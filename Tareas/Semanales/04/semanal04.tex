\documentclass[8pt, letterpaper]{article}
\usepackage[utf8]{inputenc}
\usepackage[spanish,es-nodecimaldot]{babel}
\usepackage[top=1in, bottom=1in, left=1in, right=1in]{geometry}

\usepackage{csquotes}
\usepackage{multicol}

\usepackage{qtree}

\usepackage{amsmath}
\usepackage{amsthm}
\usepackage{amssymb}
\usepackage{listings}

\title{%
  Ejercicio Semanal 4\\
  {\large{}}}
\author{Diego Méndez Medina}
\date{}
\begin{document}
\ttfamily
\maketitle
\rmfamily
Decide si los siguientes argumentos son correctos utilizando el algoritmo DPLL,
{\bf resolver el inciso 1 usando el árbol DPLL y el inciso 2 con el algoritmo
  de búsqueda de modelos.}
\begin{enumerate}
  % 01
\item Una condición necesaria para que la humanidad sea libre es que los seres
  humanos no estén ligados a una esencia. Si Dios creo a los humanos, entonces
  estamos ligados a una esencia. Es claro que los humanos somos libres. Por
  lo tanto Dios no creo a los humanos.

  \ttfamily
  {\bf Solución:}

  Utilizaremos el siguiente glosario:
  \begin{itemize}
  \item $H:$ La humanidad es libre.
  \item $E:$ Los seres humanos están ligados a un escencia.
  \item $D:$ Dios creo a los humanos.
  \end{itemize}


  \begin{multicols}{2}
    Con lo que tenemos el siguiente argumento:
  \begin{itemize}
  \item $H \rightarrow \neg E$
  \item $D \rightarrow E$
  \item $H$ \\
    \rule{.3\textwidth}{0.2mm}\\
  \item $\therefore \neg D$
  \end{itemize}
  Pasandolo a forma normal conjuntiva:
  \begin{enumerate}
  \item $fnc(H \rightarrow \neg E ) \equiv \neg H \lor \neg E$
  \item $fnc(D \rightarrow E) \equiv \neg D \lor E$
  \item $H$ \\
    \rule{.3\textwidth}{0.2mm}\\
  \item $\therefore \neg D$
  \end{enumerate}
  \end{multicols}
  Así tenemos el conjunto $S = \{\neg H\lor \neg E, \neg D\lor E, H, \neg D\}$
  
  \Tree[.$S$ [.RD$(\neg H)$ [.$\ \widehat{S}$ [.RCU$(\neg D)$ [.$\{\square\}$ ]][.RD$(E)$ [.$\ \widehat{S'}$ [.RCU$(\neg D)$ [.$\{\square\}$ ]]]]]]
    [.RCU$(H)$ [.$S'$ [.RCU$(\neg E)$ [.$S''$ [.RLP$(O)$ [.$\{\}$ ]]]] [.RD$(E)$ [.$S'''$ [.RCU$(\neg D)$ [.$\{\square\}$ ]]]]]]
  ]
  
  donde:
  \begin{align*}
    S' &= \{\neg E, \neg D\lor E, \neg D\} & S'' &= \{\neg D, \neg D\} & O &= \{\neg D, \neg D\} \\
    S''' &= \{\square, \neg D\} & \widehat{S} &= \{\neg D\lor E, \square, \neg D\} & \widehat{S'} &= \{\square, \neg D\}
  \end{align*}
  
  De forma que el argumento es correcto y existe un único modelo,  $M = \{H, \neg E\} \cup O = \{H, \neg E, \neg D\}$.
  \rmfamily
  %02
\item Si la Sra. White lo hizo, lo hizo con la llave inglesa o con la cuerda.
  Pero lo hizo con la cuerda si y sólo si el asesinato se cometió en el
  vestíbulo. El asesinato se cometió en la cocina. Por lo tanto, si la Sra.
  White lo hizo, lo hizo con la llave inglesa.

  \ttfamily
  {\bf Solución:}

  Utilizaremos el siguiente glosario:
  \begin{itemize}
  \item $W :$ La sra. White lo hizo.
  \item $I :$ La sra White lo hizo con la llave inglesa.
  \item $C :$ La sra White lo hizo con la cuerda.
  \item $V :$ El asesinato se cometió en el vestíbulo.
  \item $K :$ El asesinato se cometió en la cocina.
  \end{itemize}

  \begin{multicols}{2}
    Con lo que tenemos el siguiente argumento:
    \begin{itemize}
    \item $W \rightarrow (I \lor C)$
    \item $C \leftrightarrow V$
    \item $K$\\
      \rule{.3\textwidth}{0.2mm}\\
  \item $\therefore W \rightarrow I$
    \end{itemize}
    Pasandolo a forma normal conjuntiva:
    \begin{enumerate}
    \item $fnc(W \rightarrow (I \lor C)) \equiv \neg W \lor I \lor C$
    \item $fnc(C \leftrightarrow V) \equiv (\neg C\lor V)\land(\neg V\lor C)$
    \item $K$ \\
      \rule{.3\textwidth}{0.2mm}\\
    \item $\therefore \neg W \lor I$
    \end{enumerate}
  \end{multicols}
  Así tenemos el conjunto $\{\neg W \lor I \lor C, \neg C\lor V, \neg V\lor C, K, \neg W\lor I\}$. Comenzamos:

  \begin{align*}
    &\text{Inicio} & \cdot &\models_? \{\neg W \lor I \lor C, \neg C\lor V, \neg V\lor C, K, \neg W\lor I\}\\
    &\text{unit } K : & \cdot &\models_? \{\neg W \lor I \lor C, \neg C\lor V, \neg V\lor C, {\bf K}, \neg W\lor I\}
    \triangleright K \models_? \{\neg W \lor I \lor C, \neg C\lor V, \neg V\lor C,\neg W\lor I\} \\
    &\text{split } C: & K &\models_? \{\neg W \lor I \lor C, \neg C\lor V, \neg V\lor C,\neg W\lor I\}\\
    & & \triangleright K, C &\models_? \{\neg W \lor I \lor C, \neg C\lor V, \neg V\lor C,\neg W\lor I\} (1)\\
    & & \triangleright K, \neg C &\models_? \{\neg W \lor I \lor C, \neg C\lor V, \neg V\lor C,\neg W\lor I\} (2)
  \end{align*}
  
  Procedemos a contestar el $(1)$ si llegamos al conjunto vacío no es necesario hacer $(2)$:
  \begin{align*}
    &\text{elim }C: &  K, C &\models_? \{\neg W \lor I \lor {\bf C}, \neg C\lor V, \neg V\lor {\bf C},\neg W\lor I\}
    \triangleright K, C \models_? \{, \neg C\lor V,\neg V\lor {\bf C}, \neg W\lor I\}\\
    &\text{elim }C: & K, C &\models_? \{\neg C\lor V,\neg V\lor {\bf C}, \neg W\lor I\} \triangleright
    K, C \models_? \{, \neg C\lor V, \neg W\lor I\}\\
    &\text{red }C: & K, C &\models_? \{{\bf \neg C}\lor V, \neg W\lor I\} \triangleright K, C \models_? \{ V, \neg W\lor I\}\\
    &\text{unit }V: & K, C &\models_? \{ {\bf V}, \neg W\lor I\} \triangleright  K, C, V \models_? \{\neg W\lor I\}\\
    &\text{split }I: & K, C, V &\models_? \{\neg W\lor I\} \triangleright K, C, V, I \models_? \{\neg W\lor I\} (3) :
    \triangleright K, C, V, \neg I \models_? \{\neg W\lor I\} (4) \\
  \end{align*}
  Procedemos a resolver $(3)$, en caso de llegar a $\square$, volvemos a $(4)$:
  \begin{align*}
    &\text{elim }I: & K, C, V, I &\models_? \{\neg W\lor {\bf I}\} \triangleright K, C, V, I \models_? \{\}\\
    &\text{success }: & K, C, V, I &\models_? \{\} \text{¡Correcto!}
  \end{align*}

  Basta con mostrar un modelo para ver que es correcto, acabamos de ver que $\{ K, C, V, I\}$ es un modelo,
  así el argumento es correcto.
\end{enumerate}
\end{document}

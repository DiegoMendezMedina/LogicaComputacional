\documentclass[8pt, letterpaper]{article}
\usepackage[utf8]{inputenc}
\usepackage[spanish,es-nodecimaldot]{babel}
\usepackage[top=1in, bottom=1in, left=1in, right=1in]{geometry}

\usepackage{csquotes}
\usepackage{multicol}

\usepackage{qtree}

\usepackage{amsmath}
\usepackage{amsthm}
\usepackage{amssymb}
\usepackage{listings}

\newcommand{\m}{\mathcal{M} }
\newcommand{\I}{\mathcal{I} }

\title{%
  Ejercicio Semanal 10\\
  {\large{}}}
\author{Diego Méndez Medina}
\date{}

\begin{document}
\ttfamily
\maketitle
\rmfamily

\begin{enumerate}
\item El calendario gregoriano dice que la tierra tarda $8,700$ horas en
  rodear el sol, es decir $365\cdot 24$. Pero en realidad si lo contamos con
  un crónometro un año en realidad dura $8,675$ horas, $48$ minutos y 46
  segundos. Así que nuestro calendario se mueve un poco más rápido de lo que
  cambian nuestras estaciones del año lo cuál ocasionaría que en algunos
  años tuviéramos que celebrar navidad en medio del verano.

  Sin embargo, aunque 365 días es menos de lo necesario para representar
  de forma precisa un año, 366 son muchos más y eso aún implicaría
  un movimiento del calendario pero hacia la otra dirección. La solución
  para todo esto es el año bisiesto en donde Febrero tiene un día extra
  pero sólo cada cuatro años para que el calendario vuelva a alinearse.

  Sin embargo esto no soluciona completamente el problema, agregar
  un día al año cada $4$ años hace que el calendario avance levemente
  más lento de lo que debería. Lo que ocasiona que cada $100$ años el
  calendario se atrase un día. Lo cual podría no paracer tan grave, excepto
  que para Christopher Clavius que quería que todo se alineara perfectamente.
  Para solucionarlo cada se siglo se omite el año bisiesto, es decir, los
  años $1894$ y $1904$ fueron bisiestos pero $1900$ no.
  
  Pero aún hay un problema, el calendario ahora avanza ligeramente
  más rápido, cada $400$ años el calendario se adelantaría un día. Por
  esta razón si el siglo es divisible por $400$, entonces el año si es
  bisiesto, es decir, los años $1900$ y $2100$ no son bisiestos pero el
  $2000$ si lo fue.
  
  \begin{enumerate}
  \item Formaliza la información necesaria para saber si un año
    es bisiesto usando lógica de primer orden.

    \hfill\break
    Un año es bisiesto si cumple con lo siguiente:
    \begin{itemize}
    \item El año es divisible entre $4$.
    \item El año es un siglo y es divisible entre $400$.
    \end{itemize}

    Creamos los predicados $D^2, B^1, S^1$:
    \begin{align*}
      Dxy&: \text{$x$ divide a $y$ dejando residuo cero} \\
      Bx &: \text{$x$ es un año bisiesto}\\
      Sx &: \text{$x$ es un siglo}\\
    \end{align*}

    Y la función $f^1$:

    \begin{align*}
      fx &: \text{el número que representa al año $x$}
    \end{align*}
    
    Información necesaria para saber si es bisiesto:
    \begin{align*}
      \forall x((D4fx \land \neg Sx)\rightarrow Bx)\\
      \forall x((D4fx \land Sx \land D(400,fx))\rightarrow Bx)
    \end{align*}
    \newpage
    %b
  \item Tradúcelo a un programa lógico.

    \begin{align*}
      (D4fx \land \neg Sx)\rightarrow Bx &\equiv \neg (D4fx \land \neg Sx)\lor Bx\\
      &\equiv \neg D4fx \lor Sx\lor Bx\\
      &\equiv \neg (D4fx) \lor (Sx\lor Bx)\\
      &\equiv D4fx \rightarrow (Sx\lor Bx)\\
      &\equiv (Sx\lor Bx)\leftarrow D4fx
    \end{align*}

    \begin{align*}
      (D4fx \land Sx \land D(400,fx))\rightarrow Bx &\equiv \neg(D4fx \land Sx \land D(400,fx))\lor Bx\\
      &\equiv \neg D4fx \lor\neg Sx \lor\neg D(400,fx)\lor Bx\\
      &\equiv Bx\leftarrow(D4fx \land Sx \land D(400,fx))
    \end{align*}

    Entonces el programa $\mathbb{P}_B$, que nos indica si un año es bisiesto es:
    
    \begin{align*}
      1.& & (Sx\lor Bx) &\leftarrow D4fx\\
      2.& & Bx &\leftarrow(D4fx \land Sx \land D(400,fx))
    \end{align*}
    %c
  \item Usando resolución verifica que $2000$ fue un
    año bisiesto y que $1900$ no lo fue.

    \begin{itemize}
    \item[$2000$]

      \hfill\break
      $\mathbb{P}_B\models B2000?$
      
      \begin{align*}
        1.& & (Sx\lor Bx) &\leftarrow D4fx & &\mathbb{P}_B\\
        2.& & Bx &\leftarrow(D4fx \land Sx \land D(400,fx)) & &\mathbb{P}_B\\
        3.& & S2000&\leftarrow  & &\text{Hecho, }100\cdot 20 = 2000\\
        4.& & D4f2000&\leftarrow  & &\text{Hecho, } \ 4\cdot 500 = 2000\\
        5.& & D400f2000&\leftarrow  & &\text{Hecho, } \ 400\cdot 5 = 2000\\
        6.& & &\leftarrow  B2000 & &\text{Meta}\\
        7.& & &\leftarrow(D4f2000 \land S2000 \land D(400,f2000)) & &res(2, 6, [x:=2000]))\\
        8.& & &\leftarrow(D4f2000 \land D(400,f2000)) & &res(3, 7, []))\\
        9.& & &\leftarrow D(400,f2000) & &res(4, 8, []))\\
        10& & &\leftarrow   & &res(5, 9, []))\\
      \end{align*}

      Llegamos a la vacía, entonces $2000$ si fue par.
      \newpage
    \item[$1900$]
      
      \hfill\break
      $\mathbb{P}_B\models B1900?$
      \begin{align*}
        1.& & (Sx\lor Bx) &\leftarrow D4fx & &\mathbb{P}_B\\
        2.& & Bx &\leftarrow(D4fx \land Sx \land D(400,fx)) & &\mathbb{P}_B\\
        3.& & S1900&\leftarrow   & &\text{Hecho}\\
        4.& & D4f1900&\leftarrow  & &\text{Hecho}\\
        5.& & &\leftarrow D400f1900  & &\text{Hecho}\\
        6.& & &\leftarrow  B1900 & &\text{Meta}\\
        & &&\text{ Hay dos caminos, comenzamos con el primero:}\\
        7'.& & &\leftarrow(D4f1900 \land S1900 \land D(400,f1999)) & &res(2, 6, [x:=1900]))\\
        8'.& & &\leftarrow(D4f1900 \land D(400,f1999)) & &res(3, 7', []))\\
        9'.& & &\leftarrow D(400,f1999) & &res(4, 8', []))\\
        & &&\text{No se puede hacer nada, pasamos al segundo:}\\
        7''.& & S1900 &\leftarrow D4f1900 & &res(1, 6, [x:=1900]))\\
        8''.& & S1900 &\leftarrow  & &res(4, 7'', [x:=1900]))\\
      \end{align*}

      Hicimos ambos caminos y en ninguno llegamos a la resolución, entonces
      $1900$ no fue bisiesto.
    \end{itemize}
  \end{enumerate}
    \newpage
    % 2
  \item El periodo de órbita de cada planeta del sistema solar respecto al de la tierra es el siguiente:

    \begin{table}[h]
      \begin{center}
        \begin{tabular}{| c | c |}
          \hline
          \textbf{Planeta} & \textbf{Periodo orbital en años Tierra} \\ \hline
          Mercurio & 0.2408467\\\hline
          Venus & 0.61519726\\\hline
          Tierra & 1.0\\\hline
          Marte & 1.8808158\\\hline
          Júpiter & 11.862615\\\hline
          Saturno & 29.447498\\\hline
          Urano & 84.016846\\\hline
          Neptuno & 164.79132\\\hline
        \end{tabular}
      \end{center} 
    \end{table}

    Y un año en la Tierra tiene $31, 557, 600$ segundos. De esta forma
    si una persona tiene $1, 000, 000, 000$ segundos de vida, entonces
    tiene $31.69$ años en la Tierra y tiene $131.57$ años en Mercurio.

    \begin{enumerate}
    \item Traduce la información anterior a lógica de primer orden.

      \hfill\break
      Dada la tabla tenemos las siguientes constantes/funciones, todas
      con indice cero:

      \begin{align*}
        mercurio^0 &= 0.2408467 & venus^0 &= 0.61519726 & tierra^0&=1\\
        marte^0&=1.8808158 & jupiter^0&=11.862615  & saturno^0 &=29.447498\\
        urano^0&=84.016846 &  neptuno^0 &= 164.79132
      \end{align*}

      Y tenemos los siguientes predicados:

      \begin{align*}
        Tierra(x,y) &: x = y\cdot 31,557,600 & A(x) &:
        \text{x son años en la Tierra}
        & S(x) &: \text{x son segundos en la Tierra}\\
        div(x,y,z) &: z = x/y \\
      \end{align*}

      \begin{align*}
        Mercurio(x, y) &: \text{$x$ años en la Tierra son $y$ años en Mercurio}\\
        % venus Le preseta a mama 500, 215, 
        Venus(x, y) &: \text{$x$ años en la Tierra son $y$ años en Venus}\\
        % Marte
        Marte(x, y) &: \text{$x$ años en la Tierra son $y$ años en Marte}\\
        % jupiter
        Jupiter(x, y) &: \text{$x$ años en la Tierra son $y$ años en Jupiter}\\
        % Saturno
        Saturno(x, y) &: \text{$x$ años en la Tierra son $y$ años en Saturno}\\
        % Urano
        Urano(x, y) &: \text{$x$ años en la Tierra son $y$ años en Urano}\\
        % Neptuno
        Neptuno(x, y) &: \text{$x$ años en la Tierra son $y$ años en Neptuno}\\
      \end{align*}
      Traduciendo la información anterior a lógica de primer orden:

      \begin{align*}
        \forall x.A \exists y.S\ Tierra (x, y) \\
        % mercurio
        \forall t.A\exists y\exists m(Tierra (t, y) \land div(t, mercurio(), m)
        &\rightarrow Mercurio(m, t))\\
        % venus
        \forall t.A\exists y\exists v(Tierra (t, y) \land div(t, venus(), v)
        &\rightarrow Venus(v, t))\\
        % Marte
        \forall t.A\exists y\exists m(Tierra (t, y) \land div(t, marte(), m)
        &\rightarrow Marte(m, t))\\
        % jupiter
        \forall t.A\exists y\exists j(Tierra (t, y) \land div(t, jupiter(), j)
        &\rightarrow Jupiter(j, t))\\
        % Saturno
        \forall t.A\exists y\exists s(Tierra (t, y) \land div(t, saturno(), s)
        &\rightarrow Saturno(s, t))\\
        % Urano
        \forall t.A\exists y\exists u(Tierra (t, y) \land div(t, urano(), u)
        &\rightarrow Urano(u, t))\\
        % Neptuno
        \forall t.A\exists y\exists n(Tierra (t, y) \land div(t, neptuno(), n)
        &\rightarrow Neptuno(n, t))\\
      \end{align*}
      % b
    \item Define con ella un programa lógico.

      \hfill\break
      Tenemos el programa $\mathbb{P}_A$ que nos indica nuestra edad
      en otro planeta del sistema solar, dada la nuestra en la tierra:

      \begin{align*}
        1. & & Tierra (x, y)&\leftarrow \\
        % mercurio
        2.& & Mercurio(m, t)&\leftarrow(Tierra (t, y) \land div(t, mercurio(), m)\\
        % venus
        3.& & Venus(m, t)&\leftarrow(Tierra (t, y) \land div(t, venus(), v)\\
        % Marte
        4.& & Marte(m, t)&\leftarrow(Tierra (t, y) \land div(t, marte(), m)\\
        % jupiter
        5.& &Jupiter(m, t)&\leftarrow(Tierra (t, y) \land div(t, jupiter(), j)\\
        % Saturno
        6.& & Saturno(m, t)&\leftarrow(Tierra (t, y) \land div(t, saturno(), s)\\
        % Urano
        7.& & Urano(m, t)&\leftarrow(Tierra (t, y) \land div(t, urano(), u)\\
        % Neptuno
        8.& & Neptuno(m, t)&\leftarrow(Tierra (t, y) \land div(t, neptuno(), n)
      \end{align*}
      %c
    \item Usando resolución calcula tu edad en Júpiter.

      \hfill\break
      Antes de mostrar la ejecución del programa hay que hacer observación
      de algunas cosas.

      Nací el 31 de marzo del 2000, pero para fines del ejercicio vamos a
      asumir que tengo 21 años exactos, entonces:

      \begin{align*}
        mi\_edad\_jupiter &= 21/11.862615 & segundos\_en\_Tierra &= 21\cdot
        31, 557, 600\\
        &= 1.7702673483 & &= 662709600
      \end{align*}
      
      Y por ultimo:

      \begin{align*}
       z = 21/11.862615 \leftrightarrow z = 1.7702673483
      \end{align*}
      \newpage
      $\mathbb{P}_A\models Jupiter(21, 1.7702673483)?$

      \begin{align*}
        1. & & Tierra (x, y)&\leftarrow & &\mathbb{P}_A\\
        % mercurio
        2.& & Mercurio(m, t)&\leftarrow(Tierra (t, y) \land div(t, mercurio(), m) & &\mathbb{P}_A\\
        % venus
        3.& & Venus(m, t)&\leftarrow(Tierra (t, y) \land div(t, venus(), v) & &\mathbb{P}_A\\
        % Marte
        4.& & Marte(m, t)&\leftarrow(Tierra (t, y) \land div(t, marte(), m) & &\mathbb{P}_A\\
        % jupiter
        5.& &Jupiter(m, t)&\leftarrow(Tierra (t, y) \land div(t, jupiter(), j) & &\mathbb{P}_A\\
        % Saturno
        6.& & Saturno(m, t)&\leftarrow(Tierra (t, y) \land div(t, saturno(), s)
        & &\mathbb{P}_A\\
        % Urano
        7.& & Urano(m, t)&\leftarrow(Tierra (t, y) \land div(t, urano(), u)
         & &\mathbb{P}_A\\
        % Neptuno
        8.& & Neptuno(m, t)&\leftarrow(Tierra (t, y) \land div(t, neptuno(), n)
        & &\mathbb{P}_A\\
        % Hechos
        9.& &div(21, 11.862615, 1.7702673483)&\leftarrow & &\text{Hecho}\\
        10.& &&\leftarrow  Jupiter(21, 1.7702673483)& &\text{Meta}\\
        % Ejecución
        11.& &Jupiter(21, 1.7702673483)&\leftarrow Tierra (21, y)
        & &res(5, 9, [t, j:= 21\\
          & & & & & ,\ 1.7702673483])\\
        12.& &Jupiter(21, 1.7702673483)&\leftarrow &
        &res(1, 11, [x:=21])\\
        13.& &&\leftarrow &
        &res(10, 12, [])
      \end{align*}
    \end{enumerate}
\end{enumerate}
\end{document}

\documentclass[8pt, letterpaper]{article}
\usepackage[utf8]{inputenc}
\usepackage[spanish,es-nodecimaldot]{babel}
\usepackage[top=1in, bottom=1in, left=1in, right=1in]{geometry}

\usepackage{csquotes}
\usepackage{multicol}

\usepackage{qtree}

\usepackage{amsmath}
\usepackage{amsthm}
\usepackage{amssymb}
\usepackage{listings}

\newcommand{\m}{\mathcal{M} }
\newcommand{\I}{\mathcal{I} }

\title{%
  Ejercicio Semanal 9\\
  {\large{}}}
\author{Diego Méndez Medina}
\date{}
\begin{document}
\ttfamily
\maketitle
\rmfamily

\begin{enumerate}
\item Demuestra que $\sigma(\rho\tau) = (\sigma\rho)\tau$ para las siguientes
  sustituciones calculando todas las composiciones involucradas:
  \begin{itemize}
  \item $\sigma = [x:= a, y:= f(z), z:= u]$
  \item $\rho = [u:=y, x := f(x), z:= y]$
  \item $\tau = [u:= z, v:= g(a), z:= y]$
  \end{itemize}

  {\bf Solución:}

  \begin{itemize}
  \item $\rho\tau$
  \begin{align*}
    \rho\tau &= [u:=y\tau,\, x := f(x)\tau,\, z:= y\tau, v:= g(a)]\\
    &= [u:=y,\, x := f(x),\, z:= y,\, v:= g(a)]\\
  \end{align*}
  \item $\sigma(\rho\tau)$
  \begin{align*}
    \sigma(\rho\tau) &= [x:= a(\rho\tau),\, y:= f(z)(\rho\tau),\,
      z:= u(\rho\tau),\, u:= y,\, v:= g(a)]\\
    &= [x:= a,\, y:= f(y),\,z:= y,\, u:= y,\,v:= g(a)] & &(1)
  \end{align*}
\item $\sigma\rho$
  \begin{align*}
    \sigma\rho &= [x:= a\rho,\, y:= f(z)\rho,\, z:= u\rho,\, u:= y]\\
    &= [x:= a,\, y:= f(y),\, z:= z,\, u:= y]\\
    &= [x:= a,\, y:= f(y),\, u:= y]
  \end{align*}
\item $(\sigma\rho)\tau$
  \begin{align*}
    (\sigma\rho)\tau &= [x:= a\tau,\, y:= f(y)\tau,\, u:= y\tau,\,
      v:= g(a),\, z:= y]\\
    &= [x:= a,\, y:= f(y),\, u:= y,\, v:= g(a),\, z:= y] & &(2)
  \end{align*}
  \end{itemize}
  
  Por $(1)$ y $(2)$ vemos que $\sigma(\rho\tau) = (\sigma\rho)\tau$ para las
  composiciones dadas.
  
  %% 2
\item Verificar si los siguientes conjuntos son unificables utilizando el
  algoritmo de Martello-Montanari mostrando cada paso:
  \begin{enumerate}
  \item $W = \{Pxfxgy,\, Pafgaga,\, Pyfyga\}$ con $P^3,\, f^1,\, g^1$

    Comenzamos tratando de unificar $Pxfxgy,\, Pafgaga$:
    \begin{align*}
      \{Pxfxgy = Pafgaga\}& & &\text{Entrada}\\
      \{x = a, fx =fga, gy=ga\}& & &DESC.P\\
      \{fa =fga, gy=ga\}& & &SUST[x:=a]\\
      \{a =ga, y=a\}& & &DESC\\
      \{a =ga\}& & &SUST[y:=a]\\
      X& & &DFALLA
    \end{align*}
    
    Como este par no es unificable, aunque exista una unificación entre
    cualequiera de este par y el tercero,
    es imposible hacer la unificación de los tres.
    
  \item $W = \{fwfxhz,\, fgxfxy, fgxfab\,\}$ con $f^1\,,g^1\,,h^1$

    Comenzamos tratando de unificar $fwfxhz,\, fgxfxy$

    \begin{align*}
      fwfxhz=fgxfxy& & &\text{Entrada}\\
      \{w = gx,\, x = x,\,hz = y\}& & &DESC\\
      \{w = gx,\,hz = y\}& & &ELIM\\
      \{hz = y\}& & &SUST[w:= gx]\\
      \{y=hz\}& & &SWAP\\
      \emptyset& & &SUST[y:= hz]\\
    \end{align*}
    Son unificables bajo $\mu = [w:= gx, y:= hz]$. De tal forma que:
    \begin{align*}
      |\{fwfxhz,\,fgxfxy\}[w:= gx, y:= hz]| &= |\{fgxfxhz, fgxfxhz\}|\\
      &=  |\{fgxfxhz\}| \\
      &= 1
    \end{align*}

    Solo nos falta ver que el resultado de la unificación anterior y el
    tercer elemento($fgxfxhz,\,fgxfab$) sean unificables:
    \begin{align*}
      \{fgxfxhz = fgxfab\}& &&\text{Entrada}\\
      \{gx = gx,\,x = a,\,hz = b\}& &&DESC\\
      \{x = a,\,hz = b\}& &&ELIM\\
      \{hz = b\}& &&SUST[x:=a]\\
      X&&&DFALLA
    \end{align*}
    
    El último falló por que las constantes son consideradas funciones sin
    argumentos y $h$ es de un argumento.
    
    Así, a pesar de que el primer par es unificable y de hecho
    el segundo elemento con el tercero tambíen lo son. No existe una unificación para los tres juntos.
  \end{enumerate}
\end{enumerate}
\end{document}

\documentclass[8pt, letterpaper]{article}
\usepackage[utf8]{inputenc}
\usepackage[spanish,es-nodecimaldot]{babel}
\usepackage[top=1in, bottom=1in, left=1in, right=1in]{geometry}

\usepackage{csquotes}
\usepackage{multicol}

\usepackage{qtree}

\usepackage{amsmath}
\usepackage{amsthm}
\usepackage{amssymb}
\usepackage{listings}

\title{%
  Ejercicio Semanal 6\\
  {\large{}}}
\author{Diego Méndez Medina}
\date{}
\begin{document}
\ttfamily
\maketitle
\rmfamily
Considera el siguiente lenguaje:

\hfill\break
{\bf Consonantes:} $\{a, b, c, d, e, f\}$

\hfill\break
{\bf Predicados:} $\{\text{Sobre}^2, \text{Libre}^1, \text{Rojo}^1,
\text{Blanco}^1, \text{Verde}^1\}$

\begin{enumerate}
\item Utilizando el lenguaje anterior formaliza los siguientes enunciados:
  \begin{enumerate}
    % 1.a
  \item $a$ está sobre $c$ y $d$ está sobre $f$ o sobre $c$.

    $$Sobre(a, c) \land (Sobre(d, f) \lor Sobre(d, c))$$
    % 1.b
  \item $a$ es verde pero $c$ no lo es.
    $$Verde(a) \land \neg Verde(c)$$
    % 1.c
  \item Todo esta sobre algo.
    
    Como no conocemos el universo, asumimos que se refiere a todo elemento
    de este, así:
    $$\forall x \exists y\ Sobre(x,y)$$
    % 1.d
  \item No hay nada sobre todo lo que esta libre.
    $$\forall x (Libre(x)\rightarrow \neg(\exists y Sobre(y,x)))$$
    $$\equiv \forall x (Libre(x)\rightarrow \forall y \neg Sobre(y,x))$$
    % 1.e
  \item Hay algo rojo que no está libre.
    $$\exists x(Rojo(x) \rightarrow \neg Libre(x))$$
    % 1.f
  \item Todo lo que no es verde y esta sobre $b$ es rojo.
    $$\forall x((\neg Verde(x) \land Sobre(x, b)) \rightarrow Rojo(x))$$
  \end{enumerate}
  %2:
\item Considera las siguientes interpretaciones:
  
  {\bf Interpretación} $\mathcal{I}_1:$
  \begin{itemize}
  \item $\mathcal{I}_1(a) = b_1$, $\mathcal{I}_1(b) = b_2$,
    $\mathcal{I}_1(c) = b_3$, $\mathcal{I}_1(d) = b_4$
    $\mathcal{I}_1(e) = b_5$, $\mathcal{I}_1(f) = table$
  \item $\mathcal{I}_1(Sobre) = \{<b_1, b_4>, <b_4, b_3>, <b_3, table>,
    <b_5, b_2>, <b_2, table>\}$
  \item $\mathcal{I}_1(Libre) = \{b_1, b_5\}$
  \item $\mathcal{I}_1(Verde) = \{b_4\}$
  \item $\mathcal{I}_1(Rojo) = \{b_1, b_5\}$
  \item $\mathcal{I}_1(Blanco) = \{b_3, b_2\}$
  \end{itemize}

  {\bf Interpretación} $\mathcal{I}_2:$

  \begin{itemize}
  \item $\mathcal{I}_2(a) = hat$, $\mathcal{I}_2(b) = Joe$,
    $\mathcal{I}_2(c) = bike$, $\mathcal{I}_2(d) = Jill$,
    $\mathcal{I}_2(e) = case$, $\mathcal{I}_2(f) = piso$.
  \item $\mathcal{I}_2(Sobre) = \{<hat, Joe>, <Joe, bike>, <bike, piso>,
    <Jill, case>, <case, piso>\}$
  \item $\mathcal{I}_2(Libre) = \{hat, Jill\}$
  \item $\mathcal{I}_2(Verde) = \{hat, piso\}$
  \item $\mathcal{I}_2(Rojo) = \{bike, case\}$
  \item $\mathcal{I}_2(Blanco) = \{Joe, Jill\}$
  \end{itemize}
  Para cada formula del inciso anterior decide si es verdadera en
  $\mathcal{I}_1$ y/o en $\mathcal{I}_2$.

  \begin{itemize}
  \item[] $\mathcal{I}_1$
    \begin{itemize}
    % 1.a
    \item $Sobre(a, c) \land (Sobre(d, f) \lor Sobre(d, c))$
      \begin{align*}
        &\approx Sobre(b_1, b_3) \land (Sobre(b_4, table)\lor Sobre(b_4, b_3))\\
        & & &<b_1, b_3> \notin \mathcal{I}_1(Sobre)\\
        & & &\Rightarrow \mathcal{I}_1(Sobre(b_1, b_3)) = 0\\
        &\approx 0 \land (Sobre(b_4, table)\lor Sobre(b_4, b_3))\\
        &\text{No es verdadera en } \mathcal{I}_1
      \end{align*}
      
      \hfill\break
    % 1.b
    \item $Verde(a) \land \neg Verde(c)$
      \begin{align*}
        &\approx Verde(b_1) \land \neg Verde(b_3)\\
        & & &b_1\notin \mathcal{I}_1(Verde)\\
        & & &\Rightarrow \mathcal{I}_1(Verde(b_1)) = 0\\
        &\approx 0 \land \neg Verde(c)\\
        &\text{No es verdadera en } \mathcal{I}_1
      \end{align*}
      
      \hfill\break
      % 1.c
    \item $\forall x \exists y\ Sobre(x,y)$

      \hfill\break
      $a,b,c,d$ y $f$ son los elementos del universo, lo que buscamos
      es que cada una de sus interpretaciones  aparezca como primer elemento
      de alguna tupla en $\mathcal{I}_1(Sobre)$.

      \hfill\break
      Existe un elemento que no lo cumple, $table$, entonces no es verdadera en
      $\mathcal{I}_1$.

      \hfill\break
      % 1.d
    \item $\forall x (Libre(x)\rightarrow \forall y \neg Sobre(y,x))$

      \hfill\break
      Queremos ver que todos los elementos de $\mathcal{I}_1(Libre)$
      no figuren como segundo elemento en alguna tupla de $\mathcal{I}_1(Sobre)$
      \begin{align*}
        \forall x (Libre(x)\rightarrow \forall y \neg Sobre(y,x))\\
        &\approx (Libre(b_1)\rightarrow \forall y \neg Sobre(y,b_1)) \land
        (Libre(b_5)\rightarrow \forall y \neg Sobre(y,b_5))\\
        &\text{Es verdadera en } \mathcal{I}_1
      \end{align*}

      \hfill\break
      % 1.e
    \item $\exists x(Rojo(x) \rightarrow \neg Libre(x))$

      \hfill\break
      Para que sea verdadera debe existir un elemento que sea rojo y no este
      libre. Es decir
      $\mathcal{I}_1(Libre) \cap \mathcal{I}_1(Rojo) \neq \mathcal{I}_1(Rojo)$ 


      \hfill\break
      No es el caso, entonces no es verdadera en $\mathcal{I}_1$.

      \hfill\break
      % 1.f
    \item $\forall x((\neg Verde(x) \land Sobre(x, b)) \rightarrow Rojo(x))$

      \begin{align*}
        &\approx \forall x((\neg Verde(x) \land Sobre(x, b_2)) \rightarrow
        Rojo(x))
      \end{align*}
      Solo hay un elemento que resulta estar sobre $b_2$. Quitando los
      elementos que son verdaderos por vacuidad:
      \begin{align*}
        &\approx \neg Verde(b_5) \land Sobre(b_5, b_2)) \rightarrow
        Rojo(x)
      \end{align*}
      Así es verdadera en $\mathcal{I}_1$.
    \end{itemize}
    
    % I_2
  \item[] $\mathcal{I}_2$
    \begin{enumerate}
      % 1.a
    \item[] $Sobre(a, c) \land (Sobre(d, f) \lor Sobre(d, c))$
      % 1.b
    \item[] $Verde(a) \land \neg Verde(c)$
    % 1.c
    \item[] $\forall x \exists y\ Sobre(x,y)$
      % 1.d
    \item[] $\forall x (Libre(x)\rightarrow \forall y \neg Sobre(y,x))$
      % 1.e
    \item[] $\exists x(Rojo(x) \rightarrow \neg Libre(x))$
      % 1.f
    \item[] $\forall x((\neg Verde(x) \land Sobre(x, b)) \rightarrow Rojo(x))$
    \end{enumerate}
  \end{itemize}
\end{enumerate}
\end{document}

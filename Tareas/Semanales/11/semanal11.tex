\documentclass[8pt, letterpaper]{article}
\usepackage[utf8]{inputenc}
\usepackage[spanish,es-nodecimaldot]{babel}
\usepackage[top=1in, bottom=1in, left=1in, right=1in]{geometry}

\usepackage{csquotes}
\usepackage{multicol}

\usepackage{qtree}

\usepackage{amsmath}
\usepackage{amsthm}
\usepackage{amssymb}
\usepackage{listings}

\usepackage{enumerate}
\usepackage{tikz}
\usepackage{multicol}

\newcommand{\m}{\mathcal{M} }
\newcommand{\I}{\mathcal{I} }
\newcommand{\coso}{\mathbin{\rotatebox[origin=c]{270}{$\perp$}}}

\title{%
  Ejercicio Semanal 11\\
  {\large{}}}
\author{Diego Méndez Medina}
\date{}

\begin{document}
\ttfamily
\maketitle
\rmfamily

Demuestra lo siguiente en cálculo de secuentes:
\begin{enumerate}
  %%% 01
\item $\{(r \lor p) \land q \rightarrow l,\, m \lor q \rightarrow s
  \land t,\, s \land t \land l \rightarrow r,\, p \rightarrow q\}\coso m \land
  p \rightarrow r$

  %%% 02
\item {∀x(Hx → Gx ∧ Kx), ¬∃z(F z ∧ Gz)} \coso m ∀w¬(F w ∧ Hw)
  %%% 03
\item {(¬P ∨ Q) ∧ R, Q → S} \coso c P → R → S
\end{enumerate}
\end{document}

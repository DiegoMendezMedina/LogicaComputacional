\documentclass[8pt, letterpaper]{article}
\usepackage[utf8]{inputenc}
\usepackage[spanish,es-nodecimaldot]{babel}
\usepackage[top=1in, bottom=1in, left=1in, right=1in]{geometry}

\usepackage{csquotes}
\usepackage{multicol}

\usepackage{amsmath}
\usepackage{amsthm}
\usepackage{amssymb}
\usepackage{listings}

\title{%
  Ejercicio Semanal 2\\
  {\large{}}}
\author{López Miranda Angel Mauricio\and Diego Méndez Medina}
\date{}
\begin{document}
\ttfamily
\maketitle
\rmfamily
\begin{enumerate}
\item Defina recursivamente las siguientes funciones:

  \begin{enumerate}
    
    % inciso a primera pregunta (NI)
  \item {\tt ni} que recibe una fórmula $\varphi$ y regresa el número
    de símbolos de implicación que tiene la fórmula. Por ejemplo:

    $${\tt ni} (p \land r \to \lnot (q \to r)) = 2$$
    
    \ttfamily
    {\bf Solución:}
    \hfill\break
    % firma 
    $${\tt ni} :: {\tt PROP} \rightarrow \mathbb{N}$$
    \begin{align*} 
      % atom
      {\tt ni}(\varphi) &= 0 & &\text{si $\varphi$ es atomíca}\\
      % neg
      {\tt ni}(\neg\varphi) &= {\tt ni}(\varphi) \\
      % impl
      {\tt ni}(\varphi \rightarrow \psi) &= 1 + {\tt ni}(\varphi) +
      {\tt ni}(\psi) \\
      % resto de binarios
      {\tt ni}(\varphi * \psi) &= {\tt ni}(\varphi) + {\tt ni}(\psi) &
      &\text{donde $*$} \in\{\land, \lor, \leftrightarrow\}\\
    \end{align*}

    Ejemplo:
    \begin{align*}
      {\tt ni} (p \land r \to \lnot (q \to r)) &= 1 +
      {\tt ni} (p \land r) +  {\tt ni}(\lnot (q \to r)) \\
      &= 1 + {\tt ni}(p) +  {\tt ni}(r) +  {\tt ni}(q \to r) \\
      &= 1 + 0 +  0 +  1+ {\tt ni}(q) + {\tt ni}(r) \\
      &= 2 + 0 + 0 = 2
    \end{align*}
    \rmfamily
    
    % inciso b primera pregunta (ND)
  \item {\tt nd} que recibe una fórmula $\varphi$ y regresa el número de
    símbolos de disyunción que tiene la fórmula. Por ejemplo
    
    $${\tt nd} (p \lor r \to \lnot (q \lor r) \land (s\lor\neg t)) = 3$$
    \ttfamily
    {\bf Solución:}
    \hfill\break
    % firma 
    $${\tt nt} :: {\tt PROP} \rightarrow \mathbb{N}$$
    \begin{align*}
      % atom
      {\tt nd}(\varphi) &= 0 & &\text{si $\varphi$ es atomíca}\\
      % neg
      {\tt nd}(\neg\varphi) &= {\tt nd}(\varphi) \\
      % or
      {\tt nd}(\varphi \lor \psi) &= 1 + {\tt nd}(\varphi) +
      {\tt nd}(\psi) \\
      % resto de binarios
      {\tt nd}(\varphi * \psi) &= {\tt nd}(\varphi) + {\tt nd}(\psi) &
      &\text{donde $*$} \in\{\land, \rightarrow, \leftrightarrow\}\\
    \end{align*}
    \newpage
    Ejemplo:
    \begin{align*}
      {\tt nd} (p \lor r \to \lnot (q \lor r) \land (s\lor\neg t)) &= {\tt nd} (p \lor r \to \lnot (q \lor r) \land (s\lor\neg t)) \\
      &= {\tt nd}(p \lor r) + {\tt nd}(\lnot (q \lor r) \land (s\lor\neg t)) \\
      &= 1 + {\tt nd}(p) + {\tt nd}( r) + {\tt nd}(\lnot (q \lor r)) +
      {\tt nd}(s\lor\neg t)) \\
      &= 1 + 0 + 0 + {\tt nd}(q \lor r) + 1 +{\tt nd}(s) + {\tt nd}(\neg t) \\
      &= 2 + {\tt nd}(q \lor r) + {\tt nd}(s) + {\tt nd}(\neg t) \\
      &= 2 + 1 + {\tt nd}(q) + {\tt nd}(r) + 0 + {\tt nd}(t) \\
      &= 3 + 0 +0 +0 = 3
    \end{align*}
    
    \rmfamily
    
    % inciso c primer pregunta (QI)
  \item {\tt qi} que recibe una fórmula y regresa una fórmula en que no figura
    el símbolo $\to$, usando la equivalencia lógica
    $A\to B\equiv \neg A\lor B$. Por ejemplo:
    
    $${\tt qi} (p \land r \to \lnot (q \to r)) = \lnot (p \land r) \lor \lnot(\lnot q \lor r)$$
        \ttfamily
    {\bf Solución:}
    \hfill\break
    % firma 
    $${\tt qi} :: {\tt PROP} \rightarrow {\tt PROP}$$
    \begin{align*}
      % atom
      {\tt qi}(\varphi) &= \varphi & &\text{si $\varphi$ es atomíca}\\
      % neg
      {\tt qi}(\neg\varphi) &= \neg({\tt qi}(\varphi))\\
      % impl
      {\tt qi}(\varphi \rightarrow \psi) &= \neg({\tt qi}(\varphi))\lor {\tt qi}(\psi) \\
      % resto de binarios
      {\tt qi}(\varphi * \psi) &= {\tt qi}(\varphi) * {\tt qi}(\psi) &
      &\text{donde $*$} \in\{\land, \lor, \leftrightarrow\}
    \end{align*}
    
    Ejemplo:
    \begin{align*}
      {\tt qi} (p \land r \to \lnot (q \to r)) &= \neg({\tt qi}(p \land r))
      \lor {\tt qi}(\lnot (q \to r)) \\
      &= \neg({\tt qi}(p) \land {\tt qi}(r)) \lor \neg({\tt qi}(q \to r) \\
      &= \neg(p \land r) \lor \neg(\neg({\tt qi}(q)) \lor {\tt qi}(r))\\
      &= \neg(p \land r) \lor \neg((\neg q) \lor r)\\
    \end{align*}
  \end{enumerate}

  \newpage
  \rmfamily
  % segunda 
\item Utilizando las definiciones anteriores demuestre mediante inducción
  estructural que para cualquier fórmula $\varphi$, se cumple:
  
  $${\tt nd}({\tt qi}\,(\varphi)) = {\tt nd}(\varphi) + {\tt ni}(\varphi)$$

  \ttfamily
  {\bf Solución:}
  \hfill\break
  % Caso base
  Sea $a$ un elemento cualquiera de ATOM, dadas las definiciones
  de ni, nd y qi, la variable entra en los casos basos de las tres funciones
  y tenemos:
  \begin{align*}
    {\tt qi}(a) &= a & {\tt nd}(a) &= 0 \\
    {\tt ni}(a) &= 0
  \end{align*}
  Así:
  
  $${\tt nd}({\tt qi}(a)) = {\tt nd}(a) = 0 = 0 + 0 = {\tt nd}(a) + {\tt ni}
  (a)$$

  Como $a$ era un elemento cualquiera de ATOM, no importa si es alguna variable
  o alguna constante lógica. Para cualquiera se cumple.

  % Hipotesis de inducción
  Ahora supogamos $\varphi$ y $\psi$ son dos elementos cualesquiera de PROP
  tales que:
  \begin{align*}
    {\tt nd}({\tt qi}(\varphi)) &= {\tt nd}(\varphi) + {\tt ni}(\varphi) & (1)\\
    {\tt nd}({\tt qi}(\psi)) &= {\tt nd}(\psi) + {\tt ni}(\psi) & (2)
  \end{align*}
  
  Lo que ahora buscaremos  es constuir elementos de PROP, con $\varphi$ y
  $\psi$. ¿Como hacemos eso?, con los conectivos lógicos.
  
  %Paso inductivo.
  \hfill\break
  {\bf Negación:}

  Si construimos $\gamma$ como la negación de $\varphi$, tambien es un elemento
  de PROP. Con lo que ahora, siguiendo la construcción de nd , ni y
  $\gamma$ tenemos:
   \begin{align*}
   {\tt nd}(\gamma) &= {\tt nd}(\neg\varphi) = {\tt nd}(\varphi) & (3) \\
   {\tt ni}(\gamma) &= {\tt ni}(\neg\varphi) = {\tt ni}(\varphi) & (4)
   \end{align*}
   Entonces:
   \begin{align*}
     {\tt nd}({\tt qi}(\gamma)) &= {\tt nd}({\tt qi}(\neg\varphi)) & \text{
       Por la construcción de } \gamma \\
     &= {\tt nd}(\neg{\tt qi}(\varphi)) & \text{ Por definición de qi} \\
     &= {\tt nd}({\tt qi}(\varphi)) & \text{ Por definición de nd} \\
     \end{align*}
   Por otro lado:
   \begin{align*}
     {\tt nd}(\gamma) + {\tt ni}(\gamma) &= {\tt nd}(\varphi) +
     {\tt ni}(\varphi) & \text{Por } (3)(4)
   \end{align*}
   Así,
   $$ {\tt nd}({\tt qi}(\gamma)) = {\tt nd}({\tt qi}(\varphi)) =
   {\tt nd}(\varphi) + {\tt ni}(\varphi) = {\tt nd}(\gamma) +
   {\tt ni}(\gamma)$$
   Con lo que $\gamma = \neg\varphi$ cumple con la propiedad.

   \hfill\break
   {\bf Conjunción:}

   Ahora construimos $\gamma_1 = \varphi \land \psi$. De acuerdo a las nuestras
   definiciones de la preguna anterior tenemos:
   \begin{align*}
     {\tt nd}(\gamma_1) &= {\tt nd}(\varphi\land\psi) = {\tt nd}(\varphi) +
     {\tt nd}(\psi)& (5) \\
     {\tt ni}(\gamma_1) &= {\tt ni}(\varphi\land\psi) = {\tt ni}(\varphi) +
     {\tt ni}(\psi)& (6)
   \end{align*}
   Ahora, desarrollando:
   \begin{align*}
     {\tt nd}({\tt qi}(\gamma_1)) &= {\tt nd}({\tt qi}(\varphi\land\psi))
     & \text{
       Por la construcción de } \gamma_1 \\
     &= {\tt nd}({\tt qi}(\varphi)\land{\tt qi}(\psi)) & \text{
       Por definición de qi} \\
     &= {\tt nd}({\tt qi}(\varphi)) + {\tt nd}({\tt qi}(\psi)) &
       \text{ Por definición de nd} \\
   \end{align*}
   Seguimos desarrollando:
   \begin{align*}
     {\tt nd}({\tt qi}(\varphi)) + {\tt nd}({\tt qi}(\psi)) &=
     {\tt nd}(\varphi) + {\tt ni}(\varphi)+{\tt nd}(\psi) + {\tt ni}(\psi)
     & &\text{Por } (1)(2)\\
     &= {\tt nd}(\varphi) ++{\tt nd}(\psi) + {\tt ni}(\varphi) + 
     {\tt ni}(\psi) & &\text{Agrupando}\\
     &=  {\tt nd}(\gamma_1) + {\tt ni}(\gamma_1)& &\text{Por } (5)(6)
     \end{align*}

   Así, $\gamma_1$ cumple con la propiedad descrita:
   $${\tt nd}({\tt qi}(\gamma_1) = {\tt nd}(\gamma_1) + {\tt ni}(\gamma_1)$$
   
   \hfill\break
   {\bf Equivalecía:}
       
   Es similar a la conjunción. Construimos $\gamma_2 =
   \varphi \leftrightarrow\psi$, siguiendo nuestras definiciones:
   \begin{align*}
     {\tt nd}(\gamma_2) &= {\tt nd}(\varphi\leftrightarrow\psi) =
     {\tt nd}(\varphi) + {\tt nd}(\psi)& (7) \\
     {\tt ni}(\gamma_2) &= {\tt ni}(\varphi\leftrightarrow\psi) =
     {\tt ni}(\varphi) + {\tt ni}(\psi)& (8)
   \end{align*}
   Tenemos:
   \begin{align*}
     {\tt nd}({\tt qi}(\gamma_2)) &= {\tt nd}({\tt qi}(\varphi\leftrightarrow
     \psi))& \text{Por la construcción de } \gamma_2 \\
     &= {\tt nd}({\tt qi}(\varphi)\leftrightarrow{\tt qi}(\psi)) & \text{
       Por definición de qi} \\
     &= {\tt nd}({\tt qi}(\varphi)) + {\tt nd}({\tt qi}(\psi)) &
       \text{ Por definición de nd} \\
   \end{align*}
   Desarollamos:
   \begin{align*}
     {\tt nd}({\tt qi}(\varphi)) + {\tt nd}({\tt qi}(\psi)) &=
     {\tt nd}(\varphi) + {\tt ni}(\varphi)+{\tt nd}(\psi) + {\tt ni}(\psi)
     & &\text{Por } (1)(2)\\
     &= {\tt nd}(\varphi) + {\tt nd}(\psi) + {\tt ni}(\varphi) +
     {\tt ni}(\psi) & &\text{Agrupando}\\
     &=  {\tt nd}(\gamma_2) + {\tt ni}(\gamma_2)& &\text{Por } (7)(8)
     \end{align*}

   $\gamma_2$ cumple con la propiedad descrita:
   $${\tt nd}({\tt qi}(\gamma_2) = {\tt nd}(\gamma_2) + {\tt ni}(\gamma_2)$$
   \hfill\break
   {\bf Disyunción:}

   Ahora construimos $\gamma_3 = \varphi \lor \psi$. De acuerdo a nuestras
   respuestas al inciso anterior:
   \begin{align*}
     {\tt nd}(\gamma_3) &= {\tt nd}(\varphi\lor\psi) = 1 +
     {\tt nd}(\varphi) + {\tt nd}(\psi)& &(9) \\
     {\tt ni}(\gamma_3) &= {\tt ni}(\varphi\lor\psi) =
     {\tt ni}(\varphi) + {\tt ni}(\psi)& &(10)
   \end{align*}
   Tenemos:
   \begin{align*}
     {\tt nd}({\tt qi}(\gamma_3)) &= {\tt nd}({\tt qi}(\varphi\lor
     \psi))& \text{Por la construcción de } \gamma_3 \\
     &= {\tt nd}({\tt qi}(\varphi)\lor{\tt qi}(\psi)) & \text{
       Por definición de qi} \\
     &= 1 + {\tt nd}({\tt qi}(\varphi)) + {\tt nd}({\tt qi}(\psi)) &
       \text{ Por definición de nd} \\
   \end{align*}
   Desarollamos:
   \begin{align*}
     1 + {\tt nd}({\tt qi}(\varphi)) + {\tt nd}({\tt qi}(\psi)) &= 1 +
     {\tt nd}(\varphi) + {\tt ni}(\varphi)+{\tt nd}(\psi) + {\tt ni}(\psi)
     & &\text{Por } (1)(2)\\
     &= 1+ {\tt nd}(\varphi) + {\tt nd}(\psi) + {\tt ni}(\varphi) +
     {\tt ni}(\psi) & &\text{Agrupando}\\
     &=  {\tt nd}(\gamma_3) + {\tt ni}(\gamma_3)& &\text{Por } (9)(10)
     \end{align*}

   $\gamma_3$ cumple con la propiedad descrita:
   $${\tt nd}({\tt qi}(\gamma_3) = {\tt nd}(\gamma_3) + {\tt ni}(\gamma_3)$$
   \hfill\break
   {\bf Implicación:}
   
   Ahora construimos $\gamma_4 = \varphi \rightarrow \psi$. Siguiendo nuestras
   definiciones de ni y nd:
    \begin{align*}
      {\tt nd}(\gamma_4) &= {\tt nd}(\varphi\rightarrow\psi) = {\tt nd}
      (\varphi) + {\tt nd}(\psi)& (11) \\
      {\tt ni}(\gamma_4) &={\tt ni}(\varphi\rightarrow\psi) = 1 + {\tt ni}
      (\varphi) +
     {\tt ni}(\psi)& (12)
   \end{align*}
   Ahora, desarrollando:
   \begin{align*}
     {\tt nd}({\tt qi}(\gamma_4)) &= {\tt nd}({\tt qi}(\varphi\rightarrow\psi))
     & \text{Por la construcción de } \gamma_4 \\
     &= {\tt nd}(\neg({\tt qi}(\varphi))\lor {\tt qi}(\psi)) & \text{
       Por definición de qi} \\
     &= 1+ {\tt nd}(\neg{\tt qi}(\varphi)) + {\tt nd}({\tt qi}(\psi)) &
     \text{ Por definición de nd} \\
     &= 1+ {\tt nd}({\tt qi}(\varphi) + {\tt nd}({\tt qi}(\psi)) &
     \text{ Por definición de nd} \\
   \end{align*}
   Seguimos desarrollando:
   \begin{align*}
     1+{\tt nd}({\tt qi}(\varphi)) + {\tt nd}({\tt qi}(\psi)) &= 1+
     {\tt nd}(\varphi) + {\tt ni}(\varphi)+{\tt nd}(\psi) + {\tt ni}(\psi)
     & &\text{Por } (1)(2)\\
     &= {\tt nd}(\varphi) ++{\tt nd}(\psi) + 1 + {\tt ni}(\varphi) + 
     {\tt ni}(\psi) & &\text{Agrupando}\\
     &=  {\tt nd}(\gamma_4) + {\tt ni}(\gamma_4)& &\text{Por } (11)(12)
     \end{align*}

   Por ultimo, $\gamma_4$ cumple con la propiedad descrita:
   $${\tt nd}({\tt qi}(\gamma_4) = {\tt nd}(\gamma_4) + {\tt ni}(\gamma_4)$$

   Dados $\varphi$ y $\psi$ acabamos de construir todos los elementos
   $inmediatos$ posibles de PROP utilizando los conectivos lógicos sin repetir.

   El hecho que $\varphi$ y $\psi$ cumplieran con la propiedad implique
   que los elementos $inmediatos$ también la cumplan y que $\varphi$ y $\psi$
   eran elementos cualesquiera concluimos que para todos los elementos,
   fórmulas, de PROP se cumple:
   $${\tt nd}({\tt qi}\,(\varphi)) = {\tt nd}(\varphi) + {\tt ni}(\varphi)$$
\end{enumerate}
\end{document}

\documentclass[8pt, letterpaper]{article}
\usepackage[utf8]{inputenc}
\usepackage[spanish,es-nodecimaldot]{babel}
\usepackage[top=1in, bottom=1in, left=1in, right=1in]{geometry}

\usepackage{csquotes}
\usepackage{multicol}

\usepackage{amsmath}
\usepackage{amsthm}
\usepackage{amssymb}
\usepackage{listings}

\title{%
  Ejercicio Semanal 3\\
  {\large{}}}
\author{Diego Méndez Medina}
\date{}
\begin{document}
\ttfamily
\maketitle
\rmfamily
Considera la siguiente información:

% INFO
\begin{itemize}
\item Si Manuel es delgado, entonces Paola no es pelirroja o Fernando no es alto.
\item Si Fernando es alto, entonces Sandra es cariñosa.
\item Si Sandra es cariñosa y Paola es pelirroja, entonces Manuel es delgado.
\item Paola es pelirroja.
\end{itemize}

\begin{enumerate}
  % 01
\item Traduce cada uno de los enunciados anteriores a lógica proposicional usando el siguiente glosario:
  \begin{itemize}
  \item $M$ para indicar que Manuel es delgado.
  \item $P$ para indicar que Paola es pelirroja.
  \item $F$ para indicar que Fernando es alto.
  \item $S$ para indicar que Sandra es cariñosa.
  \end{itemize}

  \hfill\break
  \ttfamily
  %Respoesta
  {\bf Solución:}
      
  Dado el glosario antes mencionado y la información tenemos:
  \begin{itemize}
  \item $M\rightarrow (\neg P \lor \neg F)$
  \item $F\rightarrow S$
  \item $(S\land P) \rightarrow M$
  \item $P$
  \end{itemize}
  \rmfamily
  % 02
\item Encuentra la Forma Normal Negativa de las fórmula anteriores.
  
  \hfill\break
  \ttfamily
  % Respoesta 
  {\bf Solución:}
      
  Siguiendo la \textit{proposición dos de la nota tres} y la solución a la
  pregunta anterior:
  
  \begin{align*}
    M\rightarrow (\neg P \lor \neg F) &\equiv \neg M  \lor
    \neg P \lor \neg F & &\text{Eliminando implicación}\\
    &= fnn(M\rightarrow (\neg P \lor \neg F))\\
    & & &\\
    F\rightarrow S &\equiv \neg F \lor S & &\text{Eliminando implicación}\\
    &= fnn(F\rightarrow S)\\
    & & &\\
    (S\land P) \rightarrow M &\equiv \neg (S\land P) \lor M &
    &\text{Eliminando implicación}\\
    &\equiv \neg S \lor \neg P \lor M & &\text{De Morgan}\\
    &= fnn((S\land P) \rightarrow M) \\
    & & &\\
    P &= fnn(P) & &\text{Lo cumple por definición}
  \end{align*}
  \newpage
  \rmfamily
  % 03
\item Encuentra la Forma Normal Conjuntiva de cada fórmula.

  \hfill\break
  \ttfamily
  % Respoesta
  {\bf Solución:}

  Siguiendo la \textit{proposición tres de la nota tres} y la forma normal
  negativa calculada el inciso anterior, siendo bien explicitos solo en el
  primero y en los demás siguiendo lo dicho en este:
  \begin{itemize}
  \item $M\rightarrow (\neg P \lor \neg F)$
    \begin{align*}
      M\rightarrow (\neg P \lor \neg F) &\equiv \neg M  \lor
      \neg P \lor \neg F & &\text{fnn}
    \end{align*}
    $M$ es un átomico, entonces $\neg M$ es una literal. Dada la definición de
    clausula $\neg M$ es una  una clausula, sea $C_1'$.

    \hfill\break
    $P$ es un átomico, entonces $\neg P$ es una literal. Junto a la clausula
    $C_1$, $\neg P \lor C_1'$ tambíen es una clausula, llamemosla $C_2'$.

    \hfill\break
    $F$ es un átomico, $\neg F$ es una literal. Así $\neg F \lor C_2'$ es
    tambíen una clausula, sea $C_1$.

    \hfill\break
    $\top$  es un átomico y así una literal y una clausula. Y además
    $C_1 \equiv C_1 \land\top$

    \hfill\break
    Al $C_1$ y $\top$ ser clausulas, $C_1 \land \top$ es una forma normal conjuntiva. Entonces:
    $$fnc(M\rightarrow (\neg P \lor \neg F)) = C_1 \land \top=
    (\neg M  \lor\neg P \lor \neg F) \land \top \equiv \neg M
    \lor\neg P \lor\neg F$$

  \item $ F\rightarrow S$
    \begin{align*}
      F\rightarrow S &\equiv \neg F \lor S  & &\text{fnn}\\
      &\equiv (\neg F \lor S)  \land \top& &\text{juntando dos clausulas}\\
      &= fnc(F\rightarrow S)
    \end{align*}
    
  \item$(S\land P) \rightarrow M$
    \begin{align*}
      (S\land P) \rightarrow M &\equiv \neg S \lor \neg P \lor M& &\text{fnn}\\
      &\equiv (\neg S \lor \neg P \lor M) \land \top & &
      \text{juntando clausulas}\\
      &= fnc((S\land P) \rightarrow M)
    \end{align*}
    
  \item $P$
    $$P\land\top$$
  \end{itemize}
  \rmfamily
  \newpage
  % 04
\item Usando resolución binaria indica si {\it Fernando no es alto} es consecuencia lógica del conjunto de fórmulas.
  \hfill\break
  \ttfamily
  %Respoesta
 {\bf Solución:}

 Sea $\Gamma = \{M\rightarrow (\neg P \lor \neg F), F\rightarrow S,
 (S\land P) \rightarrow M,P\}$, queremos ver si:
 $$\Gamma\models \neg F$$
 Que con lo visto en clase basta demostrar que $\Gamma \cup \{\neg\neg F\}$ es
 insatisfasible, que es lo que haremos a continuación.

 \hfill\break
 El conjunto de formas normales conjuntivas para $\Gamma \cup \{\neg\neg F\}$ es:
 $$\{ \neg M  \lor\neg P \lor \neg F, \neg F \lor S, \neg S \lor \neg P \lor M,
 P, F \}$$

 De aquí obtenemos la siguiente derivación de $\square$:
 \begin{align*}
   1. \hspace{0.25cm}&\neg M  \lor\neg P \lor \neg F & &\text{Hip}\\
   2. \hspace{0.25cm}&\neg F \lor S & &\text{Hip}\\
   3. \hspace{0.25cm}&\neg S \lor \neg P \lor M & & \text{Hip}\\
   4. \hspace{0.25cm}&P & &\text{Hip}\\
   5. \hspace{0.25cm}&F & &\text{Hip}\\
   6. \hspace{0.25cm}&S & &\text{Res}(2,5)\\
   7. \hspace{0.25cm}&\neg P\lor M & &\text{Res}(3,6)\\
   8. \hspace{0.25cm}&M & &\text{Res}(4, 7)\\
   9. \hspace{0.25cm}&\neg P \lor \neg F & &\text{Res}(1,8)\\
   10. \hspace{0.25cm}&\neg F & &\text{Res}(4, 9)\\
   11. \hspace{0.25cm}&\square & &\text{Res}(5, 10)
 \end{align*}
 De manera que el argumento es correcto y {\it Fernando no es alto} es
 consecuencia de la información dada.
\end{enumerate}
\end{document}

\documentclass[8pt, letterpaper]{article}
\usepackage[utf8]{inputenc}
\usepackage[spanish,es-nodecimaldot]{babel}
\usepackage[top=1in, bottom=1in, left=1in, right=1in]{geometry}

\usepackage{csquotes}
\usepackage{multicol}

\usepackage{qtree}

\usepackage{amsmath}
\usepackage{amsthm}
\usepackage{amssymb}
\usepackage{listings}

\newcommand{\m}{\mathcal{M} }
\newcommand{\I}{\mathcal{I} }

\title{%
  Ejercicio Semanal 8\\
  {\large{}}}
\author{Diego Méndez Medina}
\date{}
\begin{document}
\ttfamily
\maketitle
\rmfamily

Decide si los siguientes argumentos lógicos son correctos o exhibe un
contraejemplo mostrando paso a paso la prueba o la construcción del
contraejemplo.
\begin{enumerate}
  %1
\item $\exists x(P(x)\land Q(x)), \exists x(P(x)\land R(x)) /
  \therefore \exists x(P(x)\land Q(x) \land R(x))$

  El argumento es incorrecto si existe un modelo $\m$ tal que:
  \begin{align*}
    \m &\models \exists x(P(x)\land Q(x))\\
    \m &\models\exists x(P(x)\land R(x))\\
    \m &\not\models \exists x(P(x)\land Q(x) \land R(x))
  \end{align*}
  
  Lo que ocurre si $\{\exists x(P(x)\land Q(x)),\ \exists x(P(x)\land R(x)),
  \ \neg \exists x(P(x)\land Q(x) \land R(x))\}$ tiene un modelo. Construyamos
  ese modelo:

  \hfill\break
  Sea $\m = <M, \I>$, tal que:
  \begin{align*}
    M &= \{a,b\} & P^\I &= \{a,b\}\\
    Q^\I &= \{a\} & R^\I &= \{b\}
  \end{align*}
  
  Antes de continuar hagamos una observación:
  \begin{align*}
    \neg \exists x(P(x)\land Q(x) \land R(x)) &\equiv
    \forall x(\neg P(x)\lor \neg Q(x) \lor\neg R(x))
  \end{align*}
  Ahora, veremos que en efecto $\m$ es modelo de
  $$\{\exists x(P(x)\land Q(x)),\
  \exists x(P(x)\land R(x)),\
  \forall x(\neg P(x)\lor \neg Q(x) \lor\neg R(x))\}$$

  \hspace*{0.5cm}Dada nuestra $\m$ antes descrita, tenemos:
  \begin{align*}
    &1.\  & \I (\exists x(P(x)\land Q(x))) &= 1 & &\text{Pues } a \in P^\I
    \text{y } a \in Q^\I\\
    &2.\  & \I (\exists x(P(x)\land R(x))) &= 1 & &\text{Pues } b \in P^\I
    \text{y } b \in R^\I\\
    &3.\  & \I_{[x/a]}(\neg P(x)\lor \neg Q(x) \lor\neg R(x))) &= 1 &
    &\text{Pues } a \notin R^\I\\
    &4.\  & \I_{[x/b]}(\neg P(x)\lor \neg Q(x) \lor\neg R(x))) &= 1 &
    &\text{Pues } b \notin Q^\I \\
    &5.\  & \I( \forall x(\neg P(x)\lor \neg Q(x) \lor\neg R(x))\}) &= 1 &
    &\text{Por $3$ y $4$}
  \end{align*}
  Así existe un modelo, $\m$, de las premisas unión la conclusión negada.
  Con lo que el argumento dado no es correcto.
  \newpage
  %2
\item $\forall x(G(x)\rightarrow H(x)), \forall x(H(x)\rightarrow F(x)),
  G(a)/\therefore \exists x(G(x) \land F(x))$

  \hfill\break
  Sea $\m$ un modelo de $\{\forall x(G(x)\rightarrow H(x)),\ 
  \forall x(H(x)\rightarrow F(x)),\ G(a)\}$, entonces:
  \begin{align*}
    \m &\models \forall x(G(x)\rightarrow H(x))\\
    \m &\models \forall(x)(H(x)\rightarrow F(x))\\
    \m &\models G(a)
  \end{align*}
  
  Queremos ver que $\m \models \exists x(G(x) \land F(x))$.

  \hfill\break
  Sea $\chi$ un estado cualquiera. tenemos:
  \begin{align*}
    1.& & \I_\chi(\forall x(G(x)\rightarrow H(x))) &= 1 & &\text{Por
      hipótesis}\\
    2.& & \I_\chi(\forall x(H(x)\rightarrow F(x))) &= 1 & &\text{Por
      hipótesis}\\
    3.& & \I_\chi(G(a)) &= 1 & &\text{Por
      hipótesis}\\
    4.& & \I_{\chi[x/a]}(G(x)\rightarrow H(x)) &= 1 & &\text{Por $1$}\\
    5.& & \I_\chi(H(a)) &= 1 & &\text{Por $3$ y $4$}\\
    6.& & \I_{\chi[x/a]}(H(x)\rightarrow F(x)) &= 1 & &\text{Por $2$}\\
    7.& & \I_\chi(F(a)) &= 1 & &\text{Por $5$ y $6$}\\
    8.& & \I_{\chi}(G(a)\land F(a)) &= 1 & &\text{Por $3$ y $7$}\\
    9.& & \I_{\chi [x/a]}(G(x)\land F(x)) &= 1 & &\text{Por $8$}\\
    10.& & \I_\chi(\exists x(G(x) \land F(x))) &= 1 & &\text{Por $9$}
  \end{align*}

  Como $\chi$ era arbitrario, concluimos que
  $\m \models \exists x(G(x) \land F(x))$, así el argumento es correcto.

  \newpage
  %3
\item \textit{Los violinistas que tocan bien son músicos de alcurnia. Hay
  algunos violinistas en la orquesta. Entonces algunos músicos son de
  alcurnia}.\rmfamily $\ (V^{(1)}, T^{(1)}, A^{(1)}, M^{(1)}, O^{(1)})$

  \hfill\break
  Definimos los predicados:
  \begin{align*}
    V(x) &= x \text{ es violinista}\\
    T(x) &= x \text{ toca bien}\\
    A(x) &= x \text{ es de alcurnia}\\
    M(x) &= x \text{ es músico}\\
    O(x) &= x \text{ está en la orquesta}\\
  \end{align*}
  
  Dados los predicados tenemos el siguiente argumento:
  \begin{align*}
    &\forall x(V(x)\land T(x)\rightarrow M(x)\land A(x)) \\
    &\exists x(V(x)\land O(x))\\
    &\text{\rule{4cm}{0.01cm}}\\
    &\therefore \exists x(M(x)\rightarrow A(x))
  \end{align*}

  Observación:
  \begin{align*}
    \neg \exists x(M(x)\rightarrow A(x)) &\equiv \forall x(\neg(\neg M(x)\lor A(x)))\\
    &\equiv \forall x(M(x)\land \neg A(x))\\
  \end{align*}

  Veremos que el argumento es incorrecto mostrando un modelo para
  $$\Sigma = \{\forall x(V(x)\land T(x)\rightarrow M(x)\land A(x)),\
  \exists x(V(x)\land O(x)),\ \forall x(M(x)\land \neg A(x))\}$$

  Sea $\m = <Mundo, \I>$, tal que:
  \begin{align*}
    Mundo &= \{ Ziggy, Leo\} & V &= \{Ziggy\} & O &= \{Ziggy, Leo\}\\
    T &= \emptyset & A &= \emptyset & M &= \{Ziggy, Leo\}
  \end{align*}

  Veamos que si satisfase a $\Sigma$:
  \begin{align*}
    \I(\forall x(V(x)\land T(x)\rightarrow M(x)\land A(x))) &= 1 &
    &\text{Por vacuidad, pues ninguno toca bien}\\
    \I(\exists x(V(x)\land O(x))) &= 1 & &\text{Pues $Ziggy\in V$ y
      $Ziggy\in O$} \\
    \I(\forall x(M(x)\land \neg A(x))) &= 1 & &\text{Pues
      $Ziggy, Leo\in M$ y $Ziggy, Leo\notin A$}
  \end{align*}
  Mostramos un modelo que satisface a las premisas y a la conclusión negada,
  con lo que el argumento es incorrecto.
\end{enumerate}
\end{document}

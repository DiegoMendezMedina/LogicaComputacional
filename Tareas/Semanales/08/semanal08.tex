\documentclass[8pt, letterpaper]{article}
\usepackage[utf8]{inputenc}
\usepackage[spanish,es-nodecimaldot]{babel}
\usepackage[top=1in, bottom=1in, left=1in, right=1in]{geometry}

\usepackage{csquotes}
\usepackage{multicol}

\usepackage{qtree}

\usepackage{amsmath}
\usepackage{amsthm}
\usepackage{amssymb}
\usepackage{listings}

\title{%
  Ejercicio Semanal 8\\
  {\large{}}}
\author{Diego Méndez Medina}
\date{}
\begin{document}
\ttfamily
\maketitle
\rmfamily

Decide si los siguientes argumentos lógicos son correctos o exhibe un
contraejemplo mostrando paso a paso la prueba o la construcción del
contraejemplo.
\begin{enumerate}
  %1
\item $\exists x(P(x)\land Q(x)), \exists x(P(x)\land R(x)) /
  \therefore \exists x(P(x)\land Q(x) \land R(x))$

  %2
\item $\forall x(G(x)\rightarrow H(x)), \forall(x)(H(x)\rightarrow F(x)),
  G(a)/\therefore \exists x(G(x) \land F(x))$
  
  
  %3
\item \textit{Los violinistas que tocan bien son músicos de alcurnia. Hay
  algunos violinistas en la orquesta. Entonces algunos músicos son de
  alcurnia}.\rmfamily $\ (V^{(1)}, T^{(1)}, A^{(1)}, M^{(1)}, O^{(1)})$
  
\end{enumerate}
\end{document}

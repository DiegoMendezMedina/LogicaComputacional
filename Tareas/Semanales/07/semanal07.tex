\documentclass[8pt, letterpaper]{article}
\usepackage[utf8]{inputenc}
\usepackage[spanish,es-nodecimaldot]{babel}
\usepackage[top=1in, bottom=1in, left=1in, right=1in]{geometry}

\usepackage{csquotes}
\usepackage{multicol}

\usepackage{qtree}

\usepackage{amsmath}
\usepackage{amsthm}
\usepackage{amssymb}
\usepackage{listings}

\title{%
  Ejercicio Semanal 7\\
  {\large{}}}
\author{Diego Méndez Medina}
\date{}
\begin{document}
\ttfamily
\maketitle
\rmfamily

Sea $\mathcal{L} = \{C^2, D^1, f^2, r^1\}$ y
$\mathcal{M}=<Lista_{NAT}, \mathcal{I}>$ donde el mundo es el conjunto
de listas finitas de números naturales y el signidicado de los simbolos es:
\begin{itemize}
\item $C^\mathcal{I} = \{ (l,l')| l'$ es la cola de $l\}$
\item $D^\mathcal{I} = \{ l| l$ tiene longitud dos $\}$
\item $r^\mathcal{I}$ calcula la reversa de $L$.
\item $f^\mathcal{I}(l,l')$ deuelve la concatenación de
  $l$ con $l'$
\end{itemize}
\begin{enumerate}
\item Para cada una de las siguientes fórmulas da un estado
  en donde se satisfaga y uno en donde no, mostrando paso a paso la apliación
  de la definición de satisfacción.
  \begin{enumerate}
  \item $D(f(x,y)) \lor C(z, r(y))$
    
    Considere los siguientes estados:
    \begin{align*}
      \mathcal{I}_\alpha(x) &= [1] & \mathcal{I}_{\alpha'}(x) &= [0,0]\\
      \mathcal{I}_\alpha(y) &= [1] & \mathcal{I}_{\alpha'}(y) &= [1, 1]\\
      \mathcal{I}_\alpha(z) &= [] & \mathcal{I}_{\alpha'}(z) &= [0,1]\\
    \end{align*}
    \begin{itemize}
      % Satisfasible
    \item[$\mathcal{I}_\alpha$]:
      Siguiendo la definición 5 de las notas del curso que se basan en la
      definición de satisfacción de Tarski:
      \begin{align*}
        \mathcal{I}_\alpha(D(f(x,y)) \lor C(z, r(y))) &= 0 & &\text{si y solo si }\mathcal{I}_\alpha(D(f,(x,y))) = \mathcal{I}_\alpha(C(z, r(y))) = 0\\
        \mathcal{I}_\alpha(f(x,y)) &= \mathcal{I}_\alpha(f([1],[1]))\\
        &= [1,1]\\
        \mathcal{I}_\alpha(D(f(x,y)) &= 1 & &\text{si y solo si }\mathcal{I}_\alpha([1,1])\in D^\mathcal{I_\alpha}\\
        \rightarrow \mathcal{I}_\alpha(D(f(x,y)) &= 1 \neq 0 \\
        \therefore \mathcal{I}_\alpha(D(f(x,y)) \lor C(z, r(y))) &= 1
      \end{align*}
      Así $D(f(x,y)) \lor C(z, r(y))$ es satisfasible en $\mathcal{M}$ pues $\mathcal{M} \models_\alpha D(f(x,y)) \lor C(z, r(y))$
      \\
      % No satisfasible
    \item[$\mathcal{I}_{\alpha'}$]:
      Siguiendo la definición de satisfacción:
      \begin{align*}
        \mathcal{I}_{\alpha'}(D(f(x,y)) \lor C(z, r(y))) &= 0 & &\text{si y solo si }\mathcal{I}_{\alpha'}(D(f,(x,y))) = \mathcal{I}_{\alpha'}(C(z, r(y))) = 0\\
        \mathcal{I}_{\alpha'}(f(x,y)) &= \mathcal{I}_{\alpha'}(f([0,0],[1, 1]))\\
        &= [0, 0, 1, 1]\\
        \mathcal{I}_{\alpha'}(D(f(x,y)) &= 1 & &\text{si y solo si }\mathcal{I}_{\alpha'}([0, 0, 1, 1])\in D^\mathcal{I_{\alpha'}}\\
        \rightarrow \mathcal{I}_{\alpha'}(D(f(x,y)) &= 0 \\
        \mathcal{I}_{\alpha'}(C(z, r(y))) &= 1 & &\text{si y solo si }\mathcal{I}_{\alpha'}([0,1],[1,1])\in C^\mathcal{I_{\alpha'}}\\
        [1] &\neq [1,1]\\
        \rightarrow \mathcal{I}_{\alpha'}(C(z, r(y))) &= 0\\
        \therefore \mathcal{I}_{\alpha'}(D(f(x,y)) \lor C(z, r(y))) &= 0
      \end{align*}
      Así $D(f(x,y)) \lor C(z, r(y))$ bajo $\mathcal{I}_{\alpha'}$ no es satisfasible en $\mathcal{M}$.
    \end{itemize}
  \item $\exists y(f(x,y) = f(y,x)\land D(r(x)))$

    \hfill\break
    Sea $\psi = f(x,y) = f(y,x)\land D(r(x))$. Considere los siguientes estados y constantes:
    $$\{x,z,w\}\in \mathcal{M}$$
    \begin{align*}
      \mathcal{I}_\chi(x) &= [0, 0] & \mathcal{I}_{\chi'}(x) &= [0,0,0]\\
      \mathcal{I}_\chi(z) &= [0] & \mathcal{I}_{\chi'}(z) &= []\\
      \mathcal{I}_\chi(w) &= [2] & \mathcal{I}_{\chi'}(w) &= [1]\\
    \end{align*}
        \begin{itemize}
      % Satisfasible
    \item[$\mathcal{I}_\chi$]:
      Siguiendo el método del inciso anterior:
      
      \begin{align*}
        \mathcal{I}_\chi(\exists y(f(x,y) = f(y,x)\land D(r(x)))) &= 1 & &\text{si y solo si }\mathcal{I}_\chi[y/m](\psi) = 1 \text{ para alguna } m \in \mathcal{M}\\
        \text{Intentemos con }\mathcal{I}_\chi[y/z](\psi)&:\\
        \mathcal{I}_\chi(f(x,z) = f(z,x)\land D(r(x))) &= 1 & &\text{si y solo si } \mathcal{I}_\chi(f(x,z) = f(z,x)) = \mathcal{I}_\chi(D(r(x)) = 1\\
        \mathcal{I}_\chi(f(x,z) = f(z,x)) &= 1 & &\text{si y solo si } f(x,z) = f(z,x)\\
        f(x,z) &= [0,0,0]\\
        &= f(z, x)\\
        \rightarrow \mathcal{I}_\chi(f(x,z) = f(z,x)) &= 1\\
        \mathcal{I}_\chi(D(r(x)) &= 1 & &\text{ si y solo si } \mathcal{I}_\chi(r(x))\in D^{\mathcal{I}_{chi}}\\
        \mathcal{I}_\chi(r(x)) &= [0,0] \\
        \rightarrow \mathcal{I}_\chi(D(r(x)) &= 1\\
        \therefore \mathcal{I}_\chi(\exists y(f(x,y) = f(y,x)\land D(r(x)))) &= 1
      \end{align*}
      Así $\exists y(f(x,y) = f(y,x)\land D(r(x)))$ es satisfasible en $\mathcal{M}$ pues:
      $$\mathcal{M} \models_\chi \exists y(f(x,y) = f(y,x)\land D(r(x)))$$
      \\
      % No satisfasible
    \item[$\mathcal{I}_{\chi'}$]:
      Siguiendo la definición de satisfacción:
      \begin{align*}
        \mathcal{I}_{\chi'}(\exists y(f(x,y) = f(y,x)\land D(r(x)))) &= 1 & &\text{si y solo si }\mathcal{I}_{\chi'}[y/m](\psi) = 1 \text{ para alguna } m \in \mathcal{M}\\
        \text{Intentemos con }\mathcal{I}_{\chi'}[y/z](\psi)&:\\
        \mathcal{I}_{\chi'}(f(x,z) = f(z,x)\land D(r(x))) &= 1 & &\text{si y solo si } \mathcal{I}_{\chi'}(f(x,z) = f(z,x)) = \mathcal{I}_{\chi'}(D(r(x)) = 1\\
        \mathcal{I}_{\chi'}(D(r(x)) &= 1 & &\text{ si y solo si } \mathcal{I}_{\chi'}(r(x))\in D^{\mathcal{I}_{\chi'}}\\
        \mathcal{I}_\chi(r(x)) &= [0,0,0] \\
        \rightarrow \mathcal{I}_\chi(D(r(x)) &= 0\\
        % segundo cambio
        \text{Intentemos con }\mathcal{I}_{\chi'}[y/w](\psi)&:\\
        \mathcal{I}_{\chi'}(f(x,w) = f(w,x)\land D(r(x))) &= 1 & &\text{si y solo si } \mathcal{I}_{\chi'}(f(x,w) = f(w,x)) = \mathcal{I}_{\chi'}(D(r(x)) = 1\\
        \mathcal{I}_{\chi'}(D(r(x)) &= 1 & &\text{ si y solo si } \mathcal{I}_{\chi'}(r(x))\in D^{\mathcal{I}_{\chi'}}\\
        \mathcal{I}_\chi(r(x)) &= [0,0,0] \\
        \rightarrow \mathcal{I}_\chi(D(r(x)) &= 0\\
        % tercer cambio
        \text{Por último, intentemos con }\mathcal{I}_{\chi'}[y/x](\psi)&:\\
        \mathcal{I}_{\chi'}(f(x,x) = f(x,x)\land D(r(x))) &= 1 & &\text{si y solo si } \mathcal{I}_{\chi'}(f(x,x) = f(x,x)) = \mathcal{I}_{\chi'}(D(r(x)) = 1\\
        \mathcal{I}_{\chi'}(D(r(x)) &= 1 & &\text{ si y solo si } \mathcal{I}_{\chi'}(r(x))\in D^{\mathcal{I}_{\chi'}}\\
        \mathcal{I}_\chi(r(x)) &= [0,0,0] \\
        \rightarrow \mathcal{I}_\chi(D(r(x)) &= 0\\
        \therefore \mathcal{I}_\chi(\exists y(f(x,y) = f(y,x)\land D(r(x)))) &= 0
      \end{align*}
      Así $\exists y(f(x,y) = f(y,x)\land D(r(x)))$ bajo $\mathcal{I}_{\chi'}$ no es satisfasible en $\mathcal{M}$.
    \end{itemize}
  \end{enumerate}
  \newpage
\item Decida si
  $\mathcal{M} \models \forall x\forall y(D(x)\land C(x,y)\rightarrow
  y = r(y))$
  
  Lo primero que vemos es que no figura ninguna variable libre en $ \forall x\forall y(D(x)\land C(x,y)\rightarrow
  y = r(y))$. Entonces podemos ver a $\psi = (D(x)\land C(x,y)\rightarrow y = r(y))$. De forma que, de acuerdo a la última
  propiedad de la sección 6 de las notas del curso:
  $$\mathcal{M} \models \forall x\forall y(D(x)\land C(x,y)\rightarrow y = r(y)) \iff \mathcal{M} \models \psi$$

  PD. $\mathcal{M} \models (D(x)\land C(x,y)\rightarrow y = r(y))$

  \hfill\break
  Sea $\mathcal{I}_\chi$ un estado cualquiera.

  PD. $\mathcal{M} \models_\chi (D(x)\land C(x,y)\rightarrow y = r(y))$
  
  PD. $\mathcal{I}_\chi(D(x)\land C(x,y)\rightarrow y = r(y)) = 1$

  \hfill\break
  Supogamos que $\mathcal{I}_\chi(D(x)\land C(x,y)) = 1$.

  Entonces por definición de $C$ y $D$, $x$ es una lista con dos
  elementos e $y$ una con un único elemento, que resulta ser
  el ultimo de $x$. Como $y$ tiene un único elemento
  su reveresa es ella misma, es decir $y = r(y)$.
  Por lo tanto $\mathcal{I}_\chi(D(x)\land C(x,y)\rightarrow y = r(y)) = 1$
  
  Como $\chi$ era un estado cualquiera, $y$ y $x$ constantes cualesquiera
  y que $\mathcal{I}_\chi(D(x)\land C(x,y)) = 1$ implica que
  $\mathcal{I}_\chi(y = r(y)) = 1$:
  $$\mathcal{M} \models \psi$$
  Lo que, por la propiedad enunciada al inicio de la solución, implica :
  $$\mathcal{M} \models \forall x\forall y(D(x)\land C(x,y)\rightarrow y = r(y))$$
\end{enumerate}
\end{document}

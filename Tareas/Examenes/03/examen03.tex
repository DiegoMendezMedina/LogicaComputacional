\documentclass[11pt,letterpaper]{article}
\usepackage[margin=2cm,includefoot]{geometry}
\usepackage[spanish]{babel}
\usepackage{graphicx}

\input{../01/macroslc}
\graphicspath{ {images/} }

\usepackage{hyperref}
\usepackage{csquotes}
\usepackage{multicol}

\usepackage{qtree}

\usepackage{amsmath}
\usepackage{amsthm}
\usepackage{amssymb}
\usepackage{listings}
\usepackage{pgfplots}

\title{Tarea Examen 3}
\author{Diego Méndez Medina}
\date{}


\begin{document}

\maketitle

\thispagestyle{empty}


\begin{enumerate}
  %%% 01 
\item ({\bf 1.5pts}) Transforma las siguientes fórmulas a su forma clausular, indica todas las formas normales necesarias por separado.

  \be
  %%% a
\item $\fa x\ex y\left(Pgfaxx \to \neg (Qafxz \lor Pxfx) \right) \to \ex z Qaxfz$

  Sea $\varphi$ la formula del inciso,
  comenzamos sacando su forma normal prenex:
  
  \begin{align*}
    \varphi 
    &\equiv
    \fa x\ex y\left(Pgfaxx \to \neg (Qafxz \lor Pxfx) \right) \to \ex w Qaxfw\\
    &\equiv
    \ex w(\fa x\ex y\left(Pgfaxx \to \neg (Qafxz \lor Pxfx) \right) \to Qaxfw)
    \\
    &\equiv
    \ex w(\fa v\ex y\left(Pgfavv \to \neg (Qafvz \lor Pvfv) \right) \to Qaxfw)\\
    &\equiv
    \ex w\ex v\fa y(\left(Pgfavv \to \neg (Qafvz \lor Pvfv) \right) \to Qawfw)
    = fnp(\varphi)
  \end{align*}
  
  Lo siguiente es buscar la forma normal de skolem pero en nuestra formula $z$
  no está libre. De acuerdo a las notas del curso basta con cuantificarla
  universalmente:
  $$fnp(\varphi) \equiv
  \fa z\ex w\ex v\fa y(\left(Pgfavv \to \neg (Qafvz \lor Pvfv) \right) \to Qawfw)
  $$
  Continuamos:
  \begin{align*}
    fnp(\varphi) &\equiv
    \fa z\ex v\fa y(\left(Pgfavv \to \neg (Qafvz \lor Pvfv) \right) \to Qahzfhz)\\
    &\equiv
    \fa z\fa y(\left(Pgfakzkz \to \neg (Qafkzz \lor Pkzfkz) \right) \to Qahzfhz)\\
    &\equiv
    \fa z\fa y(\neg\left(Pgfakzkz \to \neg (Qafkzz \lor Pkzfkz) \right) \lor Qahzfhz)\\    
    &\equiv
    \fa z\fa y(\neg\left(\neg Pgfakzkz \lor \neg (Qafkzz \lor Pkzfkz) \right) \lor Qahzfhz)\\    
    &\equiv
    \fa z\fa y(\left(Pgfakzkz \land (Qafkzz \lor Pkzfkz) \right) \lor Qahzfhz)\\    
    &\equiv
    \fa z\fa y(\left(Pgfakzkz \lor Qahzfhz\right) \land (Qafkzz \lor Pkzfkz\lor Qahzfhz))= fns (\varphi)
  \end{align*}

  Entonces la forma clausular de $\varphi$ es
  $$
  \left(Pgfakzkz \lor Qahzfhz\right) \land (Qafkzz \lor Pkzfkz\lor Qahzfhz)
  $$
  
  %%% b 
\item $\fa x\ex y\fa y Pagxy \land \neg \left( \fa x Qxfza \lor \fa x Qxfzb\right)$

  \hfill\break
  Sea $\varphi = \fa x\ex y\fa y Pagxy \land \neg \left( \fa x Qxfza \lor \fa x Qxfzb\right)$
  
  \begin{align*}
    \varphi &\equiv
    \fa x(\ex y\fa y Pagxy \land \neg \left(Qxfza \lor Qxfzb\right))\\
    &\equiv
    \fa x(\fa y Pagxy \land \neg \left(Qxfza \lor Qxfzb\right))\\
    &\equiv
    \fa x\ex y (Pagxy \land \neg \left(Qxfza \lor Qxfzb\right)) = fnp(\varphi)
  \end{align*}

  \begin{align*}
    fnp(\varphi) &\equiv
    \fa x(Pagxhx \land \neg \left(Qxfza \lor Qxfzb\right))\\
    &\equiv
    \fa x(Pagxhx \land \neg Qxfza \land \neg Qxfzb) = fns(\varphi)
  \end{align*}

  Entonces la forma clausular de $\varphi$ es
  $$
  Pagxhx \land \neg Qxfza \land \neg Qxfzb
  $$
  %%% c
\item $\neg \fa x \left( Pxz \lor \ex z Qxyz\right) \lor \ex y Pfay$

  \hfill\break
  Sea $\varphi = \neg \fa x \left( Pxz \lor \ex z Qxyz\right) \lor \ex y Pfay$

%  \begin{align*}
 %   \varphi &\equiv
  %  \ex x \neg\left( Pxz \lor \ex z Qxyz\right) \lor \ex y Pfay\\
  %  &\equiv
  %  \ex x \left(\neg Pxz \land \neg \ex z Qxyz\right) \lor \ex y Pfay\\
  %  &\equiv
  %  \ex x \left(\neg Pxz \land \fa z \neg Qxyz\right) \lor \ex y Pfay\\
  %  &\equiv
  %  \ex x \left(\neg Pxz \land \fa u \neg Qxyu\right) \lor \ex v Pfav\\
  %  &\equiv
  %  \ex x \fa u\left(\neg Pxz \land \neg Qxyu\right) \lor \ex v Pfav\\
  %      &\equiv
  %  \ex v (\ex x \fa u\left(\neg Pxz \land \neg Qxyu\right) \lor Pfav)\\
  %\end{align*}

  \begin{align*}
    \varphi &\equiv
    \fa x \left( Pxz \lor \ex z Qxyz\right) \rightarrow \ex y Pfay\\
    &\equiv
    \fa x \left( Pxz \lor \ex v Qxyv\right) \rightarrow \ex u Pfau\\
    &\equiv
    \fa x \ex v \left( Pxz \lor Qxyv\right) \rightarrow \ex u Pfau\\
    &\equiv
    \ex u (\fa x \ex v \left( Pxz \lor Qxyv\right) \rightarrow Pfau)\\
    &\equiv
    \ex u \ex x \fa v (\left( Pxz \lor Qxyv\right) \rightarrow Pfau) = fnp(\varphi)
  \end{align*}

  \begin{align*}
    fnp(\varphi) &\equiv
    \ex x \fa v (\left( Pxz \lor Qxyv\right) \rightarrow Pfab) \\
    &\equiv
    \fa v (\left( Pcz \lor Qcyv\right) \rightarrow Pfab) \\
    &\equiv
    \fa v (\neg \left( Pcz \lor Qcyv\right) \lor Pfab) \\
    &\equiv
    \fa v (\left(\neg Pcz \land \neg Qcyv\right) \lor Pfab) \\
    &\equiv
    \fa v (\left(\neg Pcz \lor Pfab\right)\land\left(\neg Qcyv\lor Pfab)\right) = fns(\varphi)
  \end{align*}

  Entonces la forma clausular de $\varphi$ es
  $$\left(\neg Pcz \lor Pfab\right)\land\left(\neg Qcyv\lor Pfab\right)$$
  \ee
  \newpage
  %%% 2
\item (\textbf{1.5pts}) 
Hallar todos los posibles resolventes de las siguientes dos cláusulas (donde 
$f^{(1)},h^{(1)}$). En cada caso dar el unificador correspondiente.
\[
\ba{l}
Pxx \lor \neg Pxhx \lor \neg Qxy \\ \\
Qfua \lor Puv
\ea
\]

\begin{align*}
  Pxx \lor \neg Pxhx \lor \neg Qxy &= \{Pxx, \neg Pxhx,\neg Qxy\}\\
  Qfua \lor Puv &= \{Qfua, Puv\}
\end{align*}

Vemos que en la primera clausula figura $\neg Pxhx$ y $\neg Qxy$
y por otro lado en la segunda figuran $Qfua$ y $Puv$.

Entonces tenemos dos resolventes:
\begin{align*}
  Res(\{Pxx, \neg Pxhx,\neg Qxy\}, \{Qfua, Puv\}, [u, v:= x, hx]) &= \{Pxx, \neg Qxy\}, \{Qfxa\}
\end{align*}

Aquí ya no podemos hacer nada, si bien hay una presencia de $Q$ en un
conjunto y en el otro la negación, no hay forma de
emparejar los elementos por que en uno figura $x$ y en el otro $fx$.

Si quisiesemos cambiar la $x$ el error se mandtendria solo
con otro elemento del universo.

Vamos con la otra resolvente:

\begin{align*}
  Res(\{Pxx, \neg Pxhx,\neg Qxy\}, \{Qfua, Puv\}, [x, y:= fu, a]) &= \{Pfufu, \neg Pfuhfu, Puv\}
\end{align*}

Nos volvemos a encontrar con el mismo problema, a pesar de ser el mismo
predicado hay distinta cardinalidad en sus funciones y por lo tanto no se puede
unificar.

\newpage
%%% 3
\item ({\bf 3pts}) Considere la siguiente información:
  \begin{itemize}
  \item Cualquier objeto es rojo, verde o azul.
  \item Los objetos rojos estan a la izquierda de los objetos verdes.
  \item La palangana está a la derecha del huacal.
  \item El huacal es verde pero la palangana no.
  \end{itemize}
Realice lo siguiente:
\begin{itemize}

  %%% 3.a
\item Verifique si la palangana es azul de manera informal, es decir
  argumentando en español.

  \hfill\break
  Sabemos que:

  Todos los objetos rojos estan a la izquierda de los objetos verdes.

  La palangana está a la derecha del huacal.

  El huacal es verde pero la palangana no lo es.

  La palangana puede ser roja? No, por que esta a la derecha del huacal
  y entonces no esta a la izquieda
  de el huacal.

  Y sabemos que no es verde. Tambíen sabemos que solo
  hay tres colores, ya descartamos dos entonces afuerza es el restante
  (azul).

  %% 3.b
\item ?` Qué información implícita, es decir, diferente a las cuatro premisas
  dadas, utilizó en el argumento informal?

  Deducir que si la palangana está a la derecha del huacal, entonces
  la palanga no puede estar a la izquierda del huacal.
  
  %%% 3.c
\item Verifique lo mismo pero de manera formal mediante resolución
  binaria. Defina claramente el glosario a utilizar y muestre la transformación de cada enunciado por separado. En
  particular debe agregar las fórmulas correspondientes a la información
  implícita usada en el argumento informal.

  \hfill\break
  Nuestro universo son los objetos en un cuarto y definimos el siguiente
  glosario:

  \begin{align*}
    Rx &:= \text{$x$ es rojo}\\
    Vx &:= \text{$x$ es verde}\\
    Ax &:= \text{$x$ es azul}\\
    Ixy &:= \text{$x$ está a la izquierda de $y$}\\
    Dxy &:= \text{$x$ está a la derecha de $y$}\\
    a &:= \text{La palangana}\\
    b &:= \text{EL huacal}
  \end{align*}

  Así formalizamos la información inicial:
  \begin{itemize}
  \item Cualquier objeto es rojo, verde o azul.
    $$ \forall x (Rx\lor Vx\lor Ax)$$
  \item Los objetos rojos estan a la izquierda de los objetos verdes.
    $$ \forall x \forall y((Rx\land Vy) \rightarrow Ixy)$$
  \item La palangana está a la derecha del huacal.
    $$ Dab$$
  \item El huacal es verde pero la palangana no.
    $$ Vb \land \neg Va$$
  \item  Si $x$ está a la derecha de $y$, entonces
    $x$ no puede estar a la izquierda de $y$.
    $$ \fa x\fa y Dxy \rightarrow \neg Ixy$$ 
  \end{itemize}

  Queremos verificar que
  $$
  \{\forall x (Rx\lor Vx\lor Ax), \forall x \forall y((Rx\land Vy) \rightarrow Ixy),
  Dab, Vb\land \neg Va, \} \models Aa
  $$

  La forma clausular de las premisas y la conclusión negada es:
  $$
  \{Rx\lor Vx\lor Ax, \neg Rx\lor \neg Vy \lor Ixy,
  Dab, Vb, \neg Va, \neg Dxy\lor \neg Ixy,\neg Aa\}
  $$

  Derivación mediante resolución:
  \begin{align*}
    1.& & Rx\lor Vx\lor Ax& &  &\text{Hip.}\\
    2.& & \neg Rx\lor\neg Vy \lor Ixy& & &\text{Hip.}\\
    3.& & Dab& &  &\text{Hip.}\\
    4.& & Vb & & &\text{Hip.}\\
    5.& & \neg Va& & &\text{Hip.}\\
    6.& & \neg Dxy\lor \neg Ixy& & &\text{Hip.}\\
    7.& & \neg Aa & & &\text{Hip.}\\
    8.& & \neg Iab & & &\text{Res(3,6, [x, y:= a, b])}\\
    9.& &  \neg Ra\lor\neg Vb& & &\text{Res(2,8, [x, y:= a, b])}\\
    10.& &  \neg Ra& & &\text{Res(4,9, [])}\\
    11.& &  Va\lor Aa& & &\text{Res(1,10, [x:=a])}\\
    12.& &  Aa& & &\text{Res(5,10, [])}\\
    13.& &  \square& & &\text{Res(7,12, [])}\\
  \end{align*}

  Dado que la cláusula vacía fue obtenida podemos concluir que la consecuencia lógica original es válida.
  %%% 3.d
\item ?` Puede resolverse este problema en {\pl} ? Justifique su respuesta.

  \hfill\break
  En clase vimos que en {\pl} sólo se permiten clausulas definidas o de
  Horn. Pero tambien tiene otras limitaciónes\footnote{https://faculty.nps.edu/ncrowe/book/chap14.html}, como no aceptar hechos con una o más $\lor$. Entonces no hay forma en {\pl}
  de escribir $Rx\lor Vx\lor Ax$. Con lo que con la info dada no se podría.
\end{itemize}

\medskip
\newpage
%%% 4
\item (\textbf{2pts}) Considera el siguiente programa lógico $\P_{1}$.
\begin{verbatim}
1.  odd(s(0)).                            
2.  odd(s(s(X)):-odd(X).    
\end{verbatim}

Construya el árbol SLD (\'arbol binario) para la siguiente meta:
$ G_{1}=?-odd(s(s(s(s(s(0)))))). $

Hay que observar que el programa tiene un hecho y una única
regla, entonces no hay mucho que hacer en el arbol. Describimos
la rama de éxito para la meta $ G_{1}=?-odd(s(s(s(s(s(0)))))). $:

\begin{align*}
  1.& & odd(s(0))& & &\text{Hip.}\\
  2.& & odd(s(s(X))):- odd(X)& & &\text{Hip.}\\
  3.& & ?-odd(s(s(s(s(s(0))))))& & &\text{Meta}\\
  4.& & ?-odd(s(s(s(0))))& & &\text{SLDRes(2,3, [X:= s(s(s(0)))])}\\
  5.& & ?-odd(s(0))& & &\text{SLDRes(2,4, [X:= s(0)])}\\
  6.& & \square & & &\text{SLDRes(1, 5, [])}\\  
\end{align*}

Entonces creo que el arbol se veria algo así:

\begin{center}
  \includegraphics[scale=0.25]{4}
\end{center}

(No pude hacer que las aristas fueran para arriba, una disculpa).


\newpage
%%% 5
\item (\textbf{2pts}) 
Considera el siguiente programa lógico $\P$:
\begin{verbatim}
1.  s(a).
2.  s(b).
3.  p(U) :- q(U,W),r(W).
4.  p(Z) :- q(Z,Z).
5.  q(X,X):- s(X).
4.  r(L):- q(L,R).
\end{verbatim}
Muestre el árbol de búsqueda para la meta $G =?.-p(Y)$.

\hfill\break
 %% ARBOL
Tuve problemas para ponerle texto a las aristas, trate
de hacerlos con tikz como en los examenes pasados
pero no pude. Si tienen alguna sugerencía por favor
comentenla aquí. De nuevo una disculpa, les adjunto imagen:

\begin{center}
  \includegraphics[scale=0.25]{5}
\end{center}

\item ({\bf Extra:
    hasta 2pts.}) Verifique la validez del siguiente argumento mediante
  resolución binaria, donde $f^{(2)},g^{(1)}$:
\[
\ba{l}
Lfxygz \lor \neg Lyz \\
\neg L fxfcfdaw \\
\hline 
\therefore \neg Lab 

\ea
\]

\item ({\bf Rescate del parcial 2 (hasta 2 puntos)}) Sean $\mathcal{L}=\{P^{(2)},\;R^{(1)},\;h^{(1)}\}$ y $\mathcal{M}=\pt{M,\I}$ donde $M=\{a,b,c\}$ y
\bi
\item[] $\mathcal{P}^\I=\{(a,b),(a,a),(b,b),(c,b)\}$
\item[] $\mathcal{R}^\I=\{a,c\}$
\item[] $h^\I(a)=b,\;h^\I(b)=b,\;h^\I(c)=a$
\ei
\begin{enumerate}
\item Decidir si  $\M\models_\sigma \fa x( Pxy \to \neg Rhx)$  donde $\sigma(y)=a $
\item Hallar un estado $\sigma$ tal que $\M\models_\sigma Pxhx \land (Rx\to Ry) $
\item Decidir si $\M\models \fa x ( Rx \to \exists y (Pxhy \land Pxy))$
\end{enumerate}

\end{enumerate}


\end{document}

\documentclass[a4paper]{article} 
\input{head}
\begin{document}

%-------------------------------
%	TITLE SECTION
%-------------------------------

\fancyhead[C]{}
\hrule \medskip % Upper rule
\begin{minipage}{0.295\textwidth} 
\raggedright
\footnotesize
\begin{figure}[H]
    \includegraphics[scale = 0.053]{style/UNAM.png}
\end{figure}
%YOURNAME \hfill\\   
%YOURSTUDENTID\hfill\\
%YOURMAIL
\end{minipage}
\begin{minipage}{0.4\textwidth} 
\centering 
\large 
Lógica Computacional 2022-I\\ 
\normalsize 
Ejercicio semanal 2\\
\normalsize
López Miranda Angel Mauricio
\end{minipage}
\begin{minipage}{0.295\textwidth} 
\raggedleft
\footnotesize
\begin{wrapfigure}{}{\textwidth} %this figure will be at the right
    \includegraphics[width=0.4\textwidth]{style/ciencias.png}
\end{wrapfigure}
%\begin{figure}[H]
 %   \includegraphics[scale = 0.12]{style/ciencias.png}
%\end{figure}
%\today\hfill\\
\end{minipage}
\medskip\hrule 
\bigskip

%-------------------------------
%	CONTENTS
%-------------------------------

%\section{First Exercise}
%\blindtext
%\subsection{First Subtask}
%Some equations
%\begin{align*}
%y &=  \sum\limits_{i,k} m_i \cdot f^k \\
%x &=  
%\underset{11}{\underbrace{3 + 8}} + 5 + 7
%\end{align*}

%\subsection{Second Subtask}
%\blindtext

%\bigskip

%------------------------------------------------

%\section{Second Exercise}
%\blindtext
%\subsection{First Subtask}

%\bigskip

%------------------------------------------------

\begin{enumerate}
\item Defina recursivamente las siguientes funciones:

	\begin{enumerate}
 		\item {\tt ni} que recibe una fórmula $\varphi$ y regresa el número de símbolos de implicación que tiene la fórmula. Por ejemplo

                  $${\tt ni} (p \land r \to \lnot (q \to r)) = 2$$
                  
        Definamos {\tt ni} de forma recursiva:\\
        ${\tt ni} \varphi=\varphi$\\
        ${\tt ni} \neg\varphi=$ ${\tt ni}(\varphi)$\\
        ${\tt ni} \vartheta \land\varphi=$ ${\tt ni} (\vartheta)+$ ${\tt ni}(\varphi)$\\
        ${\tt ni} \vartheta \lor\varphi=$ ${\tt ni} (\vartheta)+$ ${\tt ni}(\varphi)$\\
        ${\tt ni} \vartheta \to \varphi=$ $1+$ ${\tt ni} (\vartheta)+$ ${\tt ni}(\varphi)$\\
        ${\tt ni} \vartheta \leftrightarrow\varphi=$ ${\tt ni} (\vartheta)+$ ${\tt ni}(\varphi)$\\
        
		\item {\tt nd} que recibe una fórmula $\varphi$ y regresa el número de símbolos de disyunción que tiene la fórmula. Por ejemplo

                  $${\tt nd} (p \lor r \to \lnot (q \lor r) \land (s\lor\neg t)) = 3$$
                  
        Definamos {\tt nd} de forma recursiva:\\
        ${\tt nd} \varphi=\varphi$\\
        ${\tt nd} \neg\varphi=$ ${\tt nd}(\varphi)$\\
        ${\tt nd} \vartheta \land\varphi=$ ${\tt nd} (\vartheta)+$ ${\tt nd}(\varphi)$\\
        ${\tt nd} \vartheta \lor\varphi=$ $1+$ ${\tt nd} (\vartheta)+$ ${\tt nd}(\varphi)$\\
        ${\tt nd} \vartheta \to \varphi=$ ${\tt nd} (\vartheta)+$ ${\tt nd}(\varphi)$\\
        ${\tt nd} \vartheta \leftrightarrow\varphi=$ ${\tt nd} (\vartheta)+$ ${\tt nd}(\varphi)$\\
		\item {\tt qi} que recibe una fórmula y regresa una fórmula en que no figura el símbolo $\to$, usando la equivalencia lógica $A\to B\equiv \neg A\lor B$. Por ejemplo:

		$${\tt qi} (p \land r \to \lnot (q \to r)) = \lnot (p \land r) \lor \lnot(\lnot q \lor r)$$
		
		Definamos {\tt ni} de forma recursiva:\\
        ${\tt qi} \varphi=\varphi$\\
        ${\tt qi} \neg\varphi=$ ${\tt qi}(\varphi)$\\
        ${\tt qi} \vartheta \land\varphi=$ ${\tt qi} (\vartheta)\land$ ${\tt qi}(\varphi)$\\
        ${\tt qi} \vartheta \lor\varphi=$ ${\tt qi} (\vartheta)\lor$ ${\tt qi}(\varphi)$\\
        ${\tt qi} \vartheta \to \varphi=$ $\neg$ ${\tt qi} (\vartheta)\lor$ ${\tt qi}(\varphi)$\\
        ${\tt qi} \vartheta \leftrightarrow\varphi=$ ${\tt ni} (\vartheta)\leftrightarrow$ ${\tt qi}(\varphi)$\\
	\end{enumerate}
	\item Utilizando las definiciones anteriores demuestre mediante inducción estructural que para cualquier fórmula $\varphi$, se cumple:

	$${\tt nd}({\tt qi}\,(\varphi)) = {\tt nd}(\varphi) + {\tt ni}(\varphi)$$

\end{enumerate}

\end{document}

\end{document}
